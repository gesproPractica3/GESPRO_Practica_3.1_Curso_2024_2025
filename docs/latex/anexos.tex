\documentclass[a4paper,11pt,oneside]{memoir}

% Castellano
\usepackage[spanish,es-tabla]{babel}
\selectlanguage{spanish}
\usepackage[utf8]{inputenc}
\usepackage{placeins}
\usepackage{float}
\usepackage{eurosym}

\usepackage{longtable,booktabs}
\RequirePackage[table]{xcolor}
\RequirePackage{xtab}
\RequirePackage{multirow}

% Landscape
\usepackage{pdflscape}

% Multi-page tables using
\usepackage{longtable}
\usepackage{tabularx}

% Cell with line break (e.g. \specialcell{Foo\\bar})
\newcommand{\specialcell}[2][c]{%
  \begin{tabular}[#1]{@{}l@{}}#2\end{tabular}}

% Mathematic font
\usepackage{amsfonts}

% Color

% Bibliography management
\usepackage[numbers,sort]{natbib}

% Links
\usepackage[colorlinks]{hyperref}
\hypersetup{
	colorlinks,
	linkcolor={green!40!black},
	citecolor={blue!50!black},
	urlcolor={blue!80!black}
}

% Ecuaciones
\usepackage{amsmath}

% Rutas de fichero / paquete
\newcommand{\ruta}[1]{{\sffamily #1}}

% Párrafos
\nonzeroparskip

% Listas estrechas
\providecommand{\tightlist}{%
  \setlength{\itemsep}{0pt}\setlength{\parskip}{0pt}}

% Imagenes
\usepackage{graphicx}
\newcommand{\imagen}[2]{
	\begin{figure}[!h]
		\centering
		\includegraphics[width=0.9\textwidth]{#1}
		\caption{#2}\label{fig:#1}
	\end{figure}
	\FloatBarrier
}

\newcommand{\imagenAncho}[3]{
	\begin{figure}[H]
		\centering
		\includegraphics[width=#3\textwidth]{#1}
		\caption{#2}\label{fig:#1}
	\end{figure}
	\FloatBarrier
}

\newcommand{\imagenflotante}[2]{
	\begin{figure}%[!h]
		\centering
		\includegraphics[width=0.9\textwidth]{#1}
		\caption{#2}\label{fig:#1}
	\end{figure}
}



% El comando \figura nos permite insertar figuras comodamente, y utilizando
% siempre el mismo formato. Los parametros son:
% 1 -> Porcentaje del ancho de página que ocupará la figura (de 0 a 1)
% 2 --> Fichero de la imagen
% 3 --> Texto a pie de imagen
% 4 --> Etiqueta (label) para referencias
% 5 --> Opciones que queramos pasarle al \includegraphics
% 6 --> Opciones de posicionamiento a pasarle a \begin{figure}
\newcommand{\figuraConPosicion}[6]{%
  \setlength{\anchoFloat}{#1\textwidth}%
  \addtolength{\anchoFloat}{-4\fboxsep}%
  \setlength{\anchoFigura}{\anchoFloat}%
  \begin{figure}[#6]
    \begin{center}%
      \Ovalbox{%
        \begin{minipage}{\anchoFloat}%
          \begin{center}%
            \includegraphics[width=\anchoFigura,#5]{#2}%
            \caption{#3}%
            \label{#4}%
          \end{center}%
        \end{minipage}
      }%
    \end{center}%
  \end{figure}%
}

%
% Comando para incluir imágenes en formato apaisado (sin marco).
\newcommand{\figuraApaisadaSinMarco}[5]{%
  \begin{figure}%
    \begin{center}%
    \includegraphics[angle=90,height=#1\textheight,#5]{#2}%
    \caption{#3}%
    \label{#4}%
    \end{center}%
  \end{figure}%
}
% Para las tablas
\newcommand{\otoprule}{\midrule [\heavyrulewidth]}
%
% Nuevo comando para tablas pequeñas (menos de una página).
\newcommand{\tablaSmall}[5]{%
 \begin{table}[H]
  \begin{center}
   \rowcolors {2}{gray!35}{}
   \begin{tabular}{#2}
    \toprule
    #4
    \otoprule
    #5
    \bottomrule
   \end{tabular}
   \caption{#1}
   \label{tabla:#3}
  \end{center}
 \end{table}
}

%
% Nuevo comando para tablas pequeñas (menos de una página).
\newcommand{\tablaSmallSinColores}[5]{%
 \begin{table}[H]
  \begin{center}
   \begin{tabular}{#2}
    \toprule
    #4
    \otoprule
    #5
    \bottomrule
   \end{tabular}
   \caption{#1}
   \label{tabla:#3}
  \end{center}
 \end{table}
}

\newcommand{\tablaApaisadaSmall}[5]{%
\begin{landscape}
  \begin{table}
   \begin{center}
    \rowcolors {2}{gray!35}{}
    \begin{tabular}{#2}
     \toprule
     #4
     \otoprule
     #5
     \bottomrule
    \end{tabular}
    \caption{#1}
    \label{tabla:#3}
   \end{center}
  \end{table}
\end{landscape}
}

%
% Nuevo comando para tablas grandes con cabecera y filas alternas coloreadas en gris.
\newcommand{\tabla}[6]{%
  \begin{center}
    \tablefirsthead{
      \toprule
      #5
      \otoprule
    }
    \tablehead{
      \multicolumn{#3}{l}{\small\sl continúa desde la página anterior}\\
      \toprule
      #5
      \otoprule
    }
    \tabletail{
      \hline
      \multicolumn{#3}{r}{\small\sl continúa en la página siguiente}\\
    }
    \tablelasttail{
      \hline
    }
    \bottomcaption{#1}
    \rowcolors {2}{gray!35}{}
    \begin{xtabular}{#2}
      #6
      \bottomrule
    \end{xtabular}
    \label{tabla:#4}
  \end{center}
}

%
% Nuevo comando para tablas grandes con cabecera.
\newcommand{\tablaSinColores}[6]{%
  \begin{center}
    \tablefirsthead{
      \toprule
      #5
      \otoprule
    }
    \tablehead{
      \multicolumn{#3}{l}{\small\sl continúa desde la página anterior}\\
      \toprule
      #5
      \otoprule
    }
    \tabletail{
      \hline
      \multicolumn{#3}{r}{\small\sl continúa en la página siguiente}\\
    }
    \tablelasttail{
      \hline
    }
    \bottomcaption{#1}
    \begin{xtabular}{#2}
      #6
      \bottomrule
    \end{xtabular}
    \label{tabla:#4}
  \end{center}
}

%
% Nuevo comando para tablas grandes sin cabecera.
\newcommand{\tablaSinCabecera}[5]{%
  \begin{center}
    \tablefirsthead{
      \toprule
    }
    \tablehead{
      \multicolumn{#3}{l}{\small\sl continúa desde la página anterior}\\
      \hline
    }
    \tabletail{
      \hline
      \multicolumn{#3}{r}{\small\sl continúa en la página siguiente}\\
    }
    \tablelasttail{
      \hline
    }
    \bottomcaption{#1}
  \begin{xtabular}{#2}
    #5
   \bottomrule
  \end{xtabular}
  \label{tabla:#4}
  \end{center}
}



\definecolor{cgoLight}{HTML}{EEEEEE}
\definecolor{cgoExtralight}{HTML}{FFFFFF}

%
% Nuevo comando para tablas grandes sin cabecera.
\newcommand{\tablaSinCabeceraConBandas}[5]{%
  \begin{center}
    \tablefirsthead{
      \toprule
    }
    \tablehead{
      \multicolumn{#3}{l}{\small\sl continúa desde la página anterior}\\
      \hline
    }
    \tabletail{
      \hline
      \multicolumn{#3}{r}{\small\sl continúa en la página siguiente}\\
    }
    \tablelasttail{
      \hline
    }
    \bottomcaption{#1}
    \rowcolors[]{1}{cgoExtralight}{cgoLight}

  \begin{xtabular}{#2}
    #5
   \bottomrule
  \end{xtabular}
  \label{tabla:#4}
  \end{center}
}


















\graphicspath{ {../img/} }

% Capítulos
\chapterstyle{bianchi}
\newcommand{\capitulo}[2]{
	\setcounter{chapter}{#1}
	\setcounter{section}{0}
	\chapter*{#2}
	\addcontentsline{toc}{chapter}{#2}
	\markboth{#2}{#2}
}

% Apéndices
\renewcommand{\appendixname}{Apéndice}
\renewcommand*\cftappendixname{\appendixname}

\newcommand{\apendice}[1]{
	%\renewcommand{\thechapter}{A}
	\chapter{#1}
}

\renewcommand*\cftappendixname{\appendixname\ }

% Formato de portada
\makeatletter
\usepackage{xcolor}
\newcommand{\tutor}[1]{\def\@tutor{#1}}
\newcommand{\course}[1]{\def\@course{#1}}
\definecolor{cpardoBox}{HTML}{E6E6FF}
\def\maketitle{
  \null
  \thispagestyle{empty}
  % Cabecera ----------------
\noindent\includegraphics[width=\textwidth]{cabecera}\vspace{1cm}%
  \vfill
  % Título proyecto y escudo informática ----------------
  \colorbox{cpardoBox}{%
    \begin{minipage}{.8\textwidth}
      \vspace{.5cm}\Large
      \begin{center}
      \textbf{TFG del Grado en Ingeniería Informática}\vspace{.6cm}\\
      \textbf{\LARGE\@title{}}
      \end{center}
      \vspace{.2cm}
    \end{minipage}

  }%
  \hfill\begin{minipage}{.20\textwidth}
    \includegraphics[width=\textwidth]{escudoInfor}
  \end{minipage}
  \vfill
  % Datos de alumno, curso y tutores ------------------
  \begin{center}%
  {%
    \noindent\LARGE
    Presentado por \@author{}\\ 
    en la Universidad de Burgos --- \@date{}\\
    Tutores: \@tutor{}\\
  }%
  \end{center}%
  \null
  \cleardoublepage
  }
\makeatother


% Datos de portada
\title{{\Huge GoBees}\\[0.5cm]Monitorización del estado de una colmena mediante la cámara de un smartphone.}
\author{David Miguel Lozano}
\tutor{Dr. José Francisco Díez Pastor\\y Dr. Raúl Marticorena Sánchez}
\date{\today}

\begin{document}

\maketitle



\cleardoublepage



%%%%%%%%%%%%%%%%%%%%%%%%%%%%%%%%%%%%%%%%%%%%%%%%%%%%%%%%%%%%%%%%%%%%%%%%%%%%%%%%%%%%%%%%



\frontmatter


\clearpage

% Indices
\tableofcontents

\clearpage

\listoffigures

\clearpage

%\listoftables

%\clearpage

\mainmatter

\appendix

\apendice{Manuales}

\section{Introducción}\label{introduccion-plan}

La fase de planificación es un punto clave en cualquier proyecto. En
esta fase se estima el trabajo, el tiempo y el dinero que va a suponer
la realización del proyecto. Para ello, se analiza minuciosamente cada
una de las partes que componen el proyecto. Este análisis, además de
permitir conocer los recursos necesarios, es de gran ayuda en fases
posteriores del desarrollo. En este anexo se detalla todo este proceso.

La fase de planificación se puede dividir a su vez en:

\begin{itemize}
\tightlist
\item
  Planificación temporal.
\item
  Estudio de viabilidad.
\end{itemize}

En la primera parte, se elabora un calendario o un programa de tiempos.
En estos se estima el tiempo necesario para la realización de cada una
de las partes del proyecto. Se debe establecer una fecha fija de inicio
del proyecto y una fecha de finalización estimada. Teniendo en cuenta el
peso de cada una de las tareas y los requisitos que se deben cumplir
para poder empezar a trabajar en cada una de ellas.

La segunda parte se centra en la viabilidad del proyecto. El estudio de
viabilidad se puede dividir a su vez en dos apartados:

\begin{itemize}
\tightlist
\item
  Viabilidad económica: donde se estiman los costes y los beneficios que
  puede suponer la realización del proyecto.
\item
  Viabilidad legal: el contexto en el que se ejecuta el proyecto está
  regulado por una serie de leyes. Se deben analizar todas aquellas que
  afecten al proyecto. En el caso del software, las licencias y la Ley
  de Protección de Datos pueden ser los temas más relevantes.
\end{itemize}

\section{Planificación temporal}\label{planificacion-temporal}

Al inicio del proyecto se planteó utilizar una metodología ágil como
Scrum para la gestión del proyecto. Aunque no se ha seguido al 100\% la
metodología al tratarse de un proyecto educativo (no éramos un equipo de
4 a 8 personas, no hubo reuniones diarias, etc.), sí que se ha aplicado
en líneas generales una filosofía ágil:

\begin{itemize}
\tightlist
\item
  Se aplicó una estrategia de desarrollo incremental a través de
  iteraciones (\emph{sprints}) y revisiones.
\item
  La duración media de los \emph{sprints} fue de una semana.
\item
  Al finalizar cada \emph{sprint} se entregaba una parte del producto
  operativo (incremento).
\item
  Se realizaban reuniones de revisión al finalizar cada \emph{sprint} y
  al mismo tiempo de planificación del nuevo \emph{sprint}.
\item
  En la planificación del \emph{sprint} se generaba una pila de tareas a
  realizar.
\item
  Estas tareas se estimaban y priorizaban en un tablero \emph{canvas}.
\item
  Para monitorizar el progreso del proyecto se utilizó gráficos
  \emph{burndown}.
\end{itemize}

Comentar que la estimación se realizó mediante los \emph{story points}
que provee ZenHub y, a su vez, se les asignó una estimación temporal
(cota superior) que se recoge en la siguiente tabla:

\begin{longtable}[]{@{}ll@{}}
\toprule
\begin{minipage}[b]{0.22\columnwidth}\raggedright\strut
Story points\strut
\end{minipage} & \begin{minipage}[b]{0.31\columnwidth}\raggedright\strut
Estimación temporal\strut
\end{minipage}\tabularnewline
\midrule
\endhead
\begin{minipage}[t]{0.22\columnwidth}\raggedright\strut
1\strut
\end{minipage} & \begin{minipage}[t]{0.31\columnwidth}\raggedright\strut
15min\strut
\end{minipage}\tabularnewline
\begin{minipage}[t]{0.22\columnwidth}\raggedright\strut
2\strut
\end{minipage} & \begin{minipage}[t]{0.31\columnwidth}\raggedright\strut
45min\strut
\end{minipage}\tabularnewline
\begin{minipage}[t]{0.22\columnwidth}\raggedright\strut
3\strut
\end{minipage} & \begin{minipage}[t]{0.31\columnwidth}\raggedright\strut
2h\strut
\end{minipage}\tabularnewline
\begin{minipage}[t]{0.22\columnwidth}\raggedright\strut
5\strut
\end{minipage} & \begin{minipage}[t]{0.31\columnwidth}\raggedright\strut
5h\strut
\end{minipage}\tabularnewline
\begin{minipage}[t]{0.22\columnwidth}\raggedright\strut
8\strut
\end{minipage} & \begin{minipage}[t]{0.31\columnwidth}\raggedright\strut
12h\strut
\end{minipage}\tabularnewline
\begin{minipage}[t]{0.22\columnwidth}\raggedright\strut
13\strut
\end{minipage} & \begin{minipage}[t]{0.31\columnwidth}\raggedright\strut
24h\strut
\end{minipage}\tabularnewline
\begin{minipage}[t]{0.22\columnwidth}\raggedright\strut
21\strut
\end{minipage} & \begin{minipage}[t]{0.31\columnwidth}\raggedright\strut
2,5 días\strut
\end{minipage}\tabularnewline
\begin{minipage}[t]{0.22\columnwidth}\raggedright\strut
40\strut
\end{minipage} & \begin{minipage}[t]{0.31\columnwidth}\raggedright\strut
1 semana\strut
\end{minipage}\tabularnewline
\bottomrule
\caption{Equivalencia entre \emph{story points} y tiempo.}
\end{longtable}

A continuación se describen los diferentes \emph{sprints} que se han
realizado.

\subsection{Sprint 0 (09/09/16 -
16/09/16)}\label{sprint-0-090916---160916}

La reunión de planificación de este \emph{sprint} marcó el comienzo del
proyecto. En una reunión previa se había planteado la idea del proyecto
a Jose Francisco y este había aceptado tutorizarla. En esta nueva
reunión se profundizó en la idea, se incorporó Raúl Marticorena como
cotutor y se plantearon los objetivos del primer \emph{sprint}.

Los objetivos fueron: profundizar y formalizar los objetivos del
proyecto, investigar el estado del arte en algoritmos de detección y
tracking aplicados a la apicultura, establecer el conjunto de
herramientas que conformarían el entorno de desarrollo, la gestión del
proyecto y la comunicación del equipo y, por último, realizar un esquema
rápido de la aplicación que se deseaba desarrollar.

Las tareas en las que se descompusieron los objetivos se pueden ver en:
\href{https://github.com/davidmigloz/go-bees/milestone/1?closed=1}{Sprint
0}.

Se estimaron 8 horas de trabajo y se invirtieron finalmente 9,25 horas,
completando todas las tareas.

\imagen{burndowns/sprint0}{Sprint 0.}

\subsection{Sprint 1 (17/09/16 -
23/09/16)}\label{sprint-1-170916---230916}

Los objetivos de este \textbf{sprint} fueron: tomar contacto con OpenCV
para Android, realizar un curso \emph{online} de iniciación a Android,
investigar qué algoritmos de detección y tracking estaban disponibles en
OpenCV para Android y empezar a trabajar en la documentación.

En este \emph{sprint} se tuvo la suerte de hablar sobre el proyecto con
Rafael Saracchini, investigador en temas de visión artificial en el
Instituto Tecnológico de Castilla y León. Rafael nos propuso una serie
de algoritmos que nos podían ser útiles y otros que no funcionarían bajo
nuestros requisitos.

Las tareas en las que se descompusieron los objetivos se pueden ver en:
\href{https://github.com/davidmigloz/go-bees/milestone/2?closed=1}{Sprint
1}.

Se estimaron 7,25 horas de trabajo y se invirtieron finalmente 13,25
horas, completando todas las tareas.

\imagen{burndowns/sprint1}{Sprint 1.}

\subsection{Sprint 2 (24/09/16 -
29/09/16)}\label{sprint-2-240916---290916}

Los objetivos de este \emph{sprint} fueron: investigar cómo implementar
con OpenCV los algoritmos descritos en el \emph{sprint} anterior,
continuar la formación en Android y OpenCV y realizar grabaciones en el
colmenar para tener un conjunto de vídeos con los que realizar pruebas.

Las tareas en las que se descompusieron los objetivos se pueden ver en:
\href{https://github.com/davidmigloz/go-bees/milestone/3?closed=1}{Sprint
2}.

Mientras se realizaba una de las tareas del \emph{sprint}, se
encontraron dos \emph{bugs} relacionados con OpenCV y Android
(\href{https://github.com/davidmigloz/go-bees/issues/26}{\#26} y
\href{https://github.com/davidmigloz/go-bees/issues/27}{\#27}) que nos
impidieron continuar el desarrollo. El investigar su origen y buscar
soluciones supuso una gran cantidad de horas y no se lograron resolver
hasta el siguiente \emph{sprint}.

Se estimaron 11,75 horas de trabajo y se invirtieron finalmente 33
horas, quedando dos tareas pendientes para terminar durante el siguiente
\emph{sprint}.

\imagen{burndowns/sprint2}{Sprint 2.}

\subsection{Sprint 3 (30/09/16 -
06/10/16)}\label{sprint-3-300916---061016}

Los objetivos de este \emph{sprint} fueron: intentar resolver los bugs
descubiertos en el \emph{sprint} anterior, o si esto fuese imposible,
buscar una vía alternativa para continuar el proyecto y continuar
investigando las implementaciones de los algoritmos de extracción de
fondo en OpenCV.

Las tareas en las que se descompusieron los objetivos se pueden ver en:
\href{https://github.com/davidmigloz/go-bees/milestone/4?closed=1}{Sprint
3}.

Se estimaron 20,75 horas de trabajo y se invirtieron finalmente 31
horas, quedando una tarea por terminar.

\imagen{burndowns/sprint3}{Sprint 3.}

\subsection{Sprint 4 (07/10/16 -
13/10/16)}\label{sprint-4-071016---131016}

Los objetivos de este \emph{sprint} fueron: investigar técnicas de
preprocesado y potprocesado para mejorar los resultados de la fase de
extracción del fondo. Seleccionar y parametrizar el algoritmo de
extracción de fondo que provea los mejores resultados para nuestro
problema. Continuar el curso de Android. Integrar los servicios de
integración continua y documentación continua en el repositorio.

Las tareas en las que se descompusieron los objetivos se pueden ver en:
\href{https://github.com/davidmigloz/go-bees/milestone/5?closed=1}{Sprint
4}.

Se estimaron 37 horas de trabajo y se invirtieron finalmente 39,5 horas,
completando todas las tareas.

\imagen{burndowns/sprint4}{Sprint 4.}

\subsection{Sprint 5 (14/10/16 -
20/10/16)}\label{sprint-5-141016---201016}

Los objetivos de este \emph{sprint} fueron: afinar la parametrización de
los algoritmos implementados en el \emph{sprint} anterior. Detectar
contornos y contar los pertenecientes a abejas. Pensar algún método que
pueda solventar el problema del solapamiento de abejas. Documentar
\emph{sprint} anterior. Continuar la formación en Android.

Las tareas en las que se descompusieron los objetivos se pueden ver en:
\href{https://github.com/davidmigloz/go-bees/milestone/6?closed=1}{Sprint
5}.

Se estimaron 27 horas de trabajo y se invirtieron finalmente 34 horas,
completando todas las tareas.

\imagen{burndowns/sprint5}{Sprint 5.}

\subsection{Sprint 6 (21/10/16 -
27/10/16)}\label{sprint-6-211016---271016}

Los objetivos de este \emph{sprint} fueron: mudar el algoritmo de visión
artificial desarrollado en la plataforma Java a Android. Comenzar a
desarrollar una aplicación de testeo del algoritmo para conocer el error
que comete. Investigar si es posible simular el entorno de trabajo
filmando a una pantalla.

Las tareas en las que se descompusieron los objetivos se pueden ver en:
\href{https://github.com/davidmigloz/go-bees/milestone/7?closed=1}{Sprint
6}.

Mientras se mudaba el algoritmo a Android se encontró un \emph{bug} de
OpenCV (\href{https://github.com/davidmigloz/go-bees/issues/55}{\#55})
que agotaba la memoria del móvil. Este se debía a una mala liberación de
recursos por parte de OpenCV y resolvió liberándolos manualmente.

La tarea que más se desvió de su estimación fue la de testeo de los
algoritmos. Esto se debió a la dificultad añadida que supuso ejecutar
los test unitarios con dependencias de OpenCV en Travis. Finalmente, se
solventó instalando OpenCV en la máquina virtual de Travis (compilando
desde el código fuente) e inicializando la librería de forma estática
(ya que no se deseaba tener que arrancar un emulador para ejecutar los
tests unitarios).

Se estimaron 20,75 horas de trabajo y se invirtieron finalmente 41
horas, completando todas las tareas.

\imagen{burndowns/sprint6}{Sprint 6.}

\subsection{Sprint 7 (28/10/16 -
04/11/16)}\label{sprint-7-281016---041116}

Los objetivos de este \emph{sprint} fueron: estudiar patrón de
arquitectura MVP (\emph{Model-View-Presenter}) y pensar en cómo
aplicarlo al proyecto. Diseñar la posible arquitectura de la aplicación.
Estudiar el uso de inyección de dependencias en Android con Dagger 2.
Documentar las secciones de Introducción y Objetivos.

Las tareas en las que se descompusieron los objetivos se pueden ver en:
\href{https://github.com/davidmigloz/go-bees/milestone/8?closed=1}{Sprint
7}.

Se estimaron 16 horas de trabajo y se invirtieron finalmente 23 horas,
completando todas las tareas.

\imagen{burndowns/sprint7}{Sprint 7.}

\subsection{Sprint 8 (05/11/16 -
10/11/16)}\label{sprint-8-051116---101116}

Los objetivos de este \emph{sprint} fueron: diseñar el modelo de datos
de la aplicación teniendo en cuenta el uso final de estos. Desarrollar
una aplicación Java para realizar un conteo manual de un conjunto de
frames. Utilizar los datos obtenidos mediante la aplicación de conteo
para implementar un test que calcule el error que comete el algoritmo.

Las tareas en las que se descompusieron los objetivos se pueden ver en:
\href{https://github.com/davidmigloz/go-bees/milestone/9?closed=1}{Sprint
8}.

Se estimaron 46 horas de trabajo y se invirtieron finalmente 53 horas,
completando todas las tareas.

\imagen{burndowns/sprint8}{Sprint 8.}

\subsection{Sprint 9 (11/11/16 -
17/11/16)}\label{sprint-9-111116---171116}

Los objetivos de este \emph{sprint} fueron: implementar acceso a datos.
Inyección de dependencias con los \emph{build variants} de Gradle.
Empezar a desarrollar las distintas actividades de la app.

Las tareas en las que se descompusieron los objetivos se pueden ver en:
\href{https://github.com/davidmigloz/go-bees/milestone/10?closed=1}{Sprint
9}.

Se estimaron 23 horas de trabajo y se invirtieron finalmente 24,25
horas, completando todas las tareas.

\imagen{burndowns/sprint9}{Sprint 9.}

\subsection{Sprint 10 (11/11/16 -
17/11/16)}\label{sprint-10-111116---171116}

Los objetivos de este \emph{sprint} fueron: continuar desarrollando las
actividades principales de la app. Corregir documentación escrita hasta
el momento. Documentar Técnicas y herramientas y Aspectos relevantes.

Las tareas en las que se descompusieron los objetivos se pueden ver en:
\href{https://github.com/davidmigloz/go-bees/milestone/11?closed=1}{Sprint
10}.

Se estimaron 33,75 horas de trabajo y se invirtieron finalmente 39,25
horas, completando todas las tareas.

\imagen{burndowns/sprint10}{Sprint 10.}

\subsection{Sprint 11 (26/11/16 -
01/12/16)}\label{sprint-11-261116---011216}

Los objetivos de este \emph{sprint} fueron: implementar la vista detalle
de una colmena con sus grabaciones, pestañas en las vistas de colmenar y
colmena y la sección de ajustes. Corregir los errores en la
documentación indicados por los tutores. Continuar la formación en
Android.

Las tareas en las que se descompusieron los objetivos se pueden ver en:
\href{https://github.com/davidmigloz/go-bees/milestone/12?closed=1}{Sprint
11}.

Se estimaron 25,75 horas de trabajo y se invirtieron finalmente 34
horas, completando todas las tareas.

\imagen{burndowns/sprint11}{Sprint 11.}

\subsection{Sprint 12 (02/12/16 -
09/12/16)}\label{sprint-12-021216---091216}

Los objetivos de este \emph{sprint} fueron: implementar las partes de
visualización de los datos recogidos por la app (gráficos de actividad
de vuelo, temperatura, precipitaciones, vientes, etc.) Documentar
trabajos relacionados. Empezar a desarrollar la web del producto.

Las tareas en las que se descompusieron los objetivos se pueden ver en:
\href{https://github.com/davidmigloz/go-bees/milestone/13?closed=1}{Sprint
12}.

Se estimaron 36,25 horas de trabajo y se invirtieron finalmente 50,75
horas, completando todas las tareas.

\imagen{burndowns/sprint12}{Sprint 12.}

\subsection{Sprint 13 (10/12/16 -
14/12/16)}\label{sprint-13-101216---141216}

Los objetivos de este \emph{sprint} fueron: agregar opción de
localización GPS al añadir colmenar. Incluir una tabla comparativa en la
sección Trabajos relacionados.

Las tareas en las que se descompusieron los objetivos se pueden ver en:
\href{https://github.com/davidmigloz/go-bees/milestone/14?closed=1}{Sprint
13}.

Se estimaron 26,25 horas de trabajo y se invirtieron finalmente 14,25
horas, completando todas las tareas.

\imagen{burndowns/sprint13}{Sprint 13.}

\subsection{Sprint 14 (15/12/16 -
11/01/17)}\label{sprint-14-151216---110117}

Se trató del sprint más largo de todos los realizados, con una duración
de cuatro semanas debido a las vacaciones de Navidad.

Los objetivos de este \emph{sprint} fueron: implementar el servicio de
monitorización en segundo plano, junto con su sección de ajustes, la
obtención de información meteorológica, la edición y borrado de
colmenares y colmenas y las pestañas de información de colmenar y
colmena. Además, realizar un estudio de viabilidad legal y seleccionar
la licencia más apropiada para el proyecto.

Las tareas en las que se descompusieron los objetivos se pueden ver en:
\href{https://github.com/davidmigloz/go-bees/milestone/15?closed=1}{Sprint
14}.

Se estimaron 143 horas de trabajo y se invirtieron finalmente 187,75
horas, completando todas las tareas.

\imagen{burndowns/sprint14}{Sprint 14.}

\subsection{Sprint 15 (12/01/17 -
18/01/17)}\label{sprint-15-120117---180117}

Los objetivos de este \emph{sprint} fueron: finalizar el desarrollo
principal de la app completando el menú y la internacionalización.
Completar los contenidos de la memoria y continuar con los anexos ``Plan
del proyecto software'' y ``Requisitos.''

Las tareas en las que se descompusieron los objetivos se pueden ver en:
\href{https://github.com/davidmigloz/go-bees/milestone/16?closed=1}{Sprint
15}.

Se estimaron 39 horas de trabajo y se invirtieron finalmente 37,75
horas, a falta de terminar los anexos planificados por falta de tiempo.

\imagen{burndowns/sprint15}{Sprint 15.}

\subsection{Sprint 16 (19/01/17 -
25/01/17)}\label{sprint-16-190117---250117}

Los objetivos de este \emph{sprint} fueron: completar las tareas
pendientes del anterior sprint (Especificación de requisitos y Análisis
económico), documentar el diseño de datos, procedimental y
arquitectónico y aumentar la cobertura de los test.

Las tareas en las que se descompusieron los objetivos se pueden ver en:
\href{https://github.com/davidmigloz/go-bees/milestone/17?closed=1}{Sprint
16}.

Se estimaron 45,75 horas de trabajo y se invirtieron finalmente 45,25
horas, completando todas las tareas.

\imagen{burndowns/sprint16}{Sprint 16.}

\subsection{Sprint 17 (26/01/17 -
02/02/17)}\label{sprint-17-260117---020217}

Los objetivos de este \emph{sprint} fueron: continuar anexos. Convertir
la memoria a formato LaTeX. Pulir los últimos detalles de la aplicación
y publicarla en Google Play.

Las tareas en las que se descompusieron los objetivos se pueden ver en:
\href{https://github.com/davidmigloz/go-bees/milestone/18?closed=1}{Sprint
17}.

Se estimaron 53,50 horas de trabajo y se invirtieron finalmente 56,50
horas, completando todas las tareas.

\imagen{burndowns/sprint17}{Sprint 17.}

\subsection{Sprint 18 (02/02/17 -
07/02/17)}\label{sprint-18-020217---070217}

Los objetivos de este \emph{sprint} fueron: imprimir memoria, terminar
anexos y corrección de errores.

Las tareas en las que se descompusieron los objetivos se pueden ver en:
\href{https://github.com/davidmigloz/go-bees/milestone/19?closed=1}{Sprint
18}.

Se estimaron 41 horas de trabajo y se invirtieron finalmente 41 horas,
completando todas las tareas.

\imagen{burndowns/sprint18}{Sprint 18}

\subsection{Resumen}\label{resumen}

En la siguiente tabla se muestra un resumen del tiempo dedicado a los
distintos tipos de tareas.

\begin{longtable}[]{@{}lrr@{}}
\toprule
\begin{minipage}[b]{0.37\columnwidth}\raggedright\strut
Categoría\strut
\end{minipage} & \begin{minipage}[b]{0.19\columnwidth}\raggedright\strut
\emph{Issues}\strut
\end{minipage} & \begin{minipage}[b]{0.19\columnwidth}\raggedright\strut
Tiempo (h)\strut
\end{minipage}\tabularnewline
\midrule
\endhead
\begin{minipage}[t]{0.37\columnwidth}\raggedright\strut
\emph{Bug}\strut
\end{minipage} & \begin{minipage}[t]{0.19\columnwidth}\raggedright\strut
26\strut
\end{minipage} & \begin{minipage}[t]{0.19\columnwidth}\raggedright\strut
40,75\strut
\end{minipage}\tabularnewline
\begin{minipage}[t]{0.37\columnwidth}\raggedright\strut
\emph{Documentation}\strut
\end{minipage} & \begin{minipage}[t]{0.19\columnwidth}\raggedright\strut
41\strut
\end{minipage} & \begin{minipage}[t]{0.19\columnwidth}\raggedright\strut
106\strut
\end{minipage}\tabularnewline
\begin{minipage}[t]{0.37\columnwidth}\raggedright\strut
\emph{Feature}\strut
\end{minipage} & \begin{minipage}[t]{0.19\columnwidth}\raggedright\strut
63\strut
\end{minipage} & \begin{minipage}[t]{0.19\columnwidth}\raggedright\strut
410\strut
\end{minipage}\tabularnewline
\begin{minipage}[t]{0.37\columnwidth}\raggedright\strut
\emph{Research}\strut
\end{minipage} & \begin{minipage}[t]{0.19\columnwidth}\raggedright\strut
30\strut
\end{minipage} & \begin{minipage}[t]{0.19\columnwidth}\raggedright\strut
128\strut
\end{minipage}\tabularnewline
\begin{minipage}[t]{0.37\columnwidth}\raggedright\strut
\emph{Testing}\strut
\end{minipage} & \begin{minipage}[t]{0.19\columnwidth}\raggedright\strut
7\strut
\end{minipage} & \begin{minipage}[t]{0.19\columnwidth}\raggedright\strut
49\strut
\end{minipage}\tabularnewline
\midrule
\begin{minipage}[t]{0.37\columnwidth}\raggedright\strut
TOTAL\strut
\end{minipage} & \begin{minipage}[t]{0.19\columnwidth}\raggedright\strut
167\strut
\end{minipage} & \begin{minipage}[t]{0.19\columnwidth}\raggedright\strut
794\strut
\end{minipage}\tabularnewline
\bottomrule
\caption{Desglose de las horas dedicadas al proyecto.}
\end{longtable}

\imagenAncho{project-sumary}{Porcentaje de horas dedicadas por categoría.}{0.7}
\newpage
\section{Estudio de viabilidad}\label{estudio-de-viabilidad}

\subsection{Viabilidad económica}\label{viabilidad-econuxf3mica}

En el siguiente apartado se analizarán los costes y beneficios que
podría haber supuesto el proyecto si se hubiese realizado en un entorno
empresarial real.

\subsubsection{Costes}\label{costes}

La estructura de costes del proyecto se puede desglosar en las
siguientes categorías.

\textbf{Costes de personal:}

El proyecto ha sido llevado a cabo por un desarrollador empleado a
tiempo completo durante cinco meses. Se considera el siguiente salario:

\begin{longtable}[]{@{}lr@{}}
\toprule
\begin{minipage}[b]{0.38\columnwidth}\raggedright\strut
\textbf{Concepto}\strut
\end{minipage} & \begin{minipage}[b]{0.20\columnwidth}\raggedright\strut
\textbf{Coste}\strut
\end{minipage}\tabularnewline
\midrule
\endhead
\begin{minipage}[t]{0.38\columnwidth}\raggedright\strut
Salario mensual neto\strut
\end{minipage} & \begin{minipage}[t]{0.20\columnwidth}\raggedright\strut
1.000\euro{}\strut
\end{minipage}\tabularnewline
\begin{minipage}[t]{0.38\columnwidth}\raggedright\strut
Retención IRPF (15\%)\strut
\end{minipage} & \begin{minipage}[t]{0.20\columnwidth}\raggedright\strut
272,23\euro{}\strut
\end{minipage}\tabularnewline
\begin{minipage}[t]{0.38\columnwidth}\raggedright\strut
Seguridad Social (29,9\%)\strut
\end{minipage} & \begin{minipage}[t]{0.20\columnwidth}\raggedright\strut
542,65\euro{}\strut
\end{minipage}\tabularnewline
\begin{minipage}[t]{0.38\columnwidth}\raggedright\strut
Salario mensual bruto\strut
\end{minipage} & \begin{minipage}[t]{0.20\columnwidth}\raggedright\strut
1.814,88\euro{}\strut
\end{minipage}\tabularnewline
\midrule
\begin{minipage}[t]{0.38\columnwidth}\raggedright\strut
\textbf{Total 5 meses}\strut
\end{minipage} & \begin{minipage}[t]{0.20\columnwidth}\raggedright\strut
9.074,40 \euro{}\strut
\end{minipage}\tabularnewline
\bottomrule
\caption{Costes de personal.}
\end{longtable}

La retribución a la Seguridad Social se ha calculado como un 23,60\% por
contingencias comunes, más un 5,50\% por desempleo de tipo general, más
un 0,20\% para el Fondo de Garantía Salarial y más un 0,60\% de
formación profesional. En total un 29,9\% que se aplica al salario bruto
\citep{ss_cotizacion}.

\textbf{Costes de \emph{hardware}:}

En este apartado se revisan todos los costes en dispositivos
\emph{hardware} que se han necesitado para el desarrollo del proyecto.
Se considera que la amortización ronda los 5 años y han sido utilizados
durante 5 meses.

\begin{longtable}[]{@{}lrr@{}}
\toprule
\begin{minipage}[b]{0.29\columnwidth}\raggedright\strut
\textbf{Concepto}\strut
\end{minipage} & \begin{minipage}[b]{0.18\columnwidth}\raggedright\strut
\textbf{Coste}\strut
\end{minipage} & \begin{minipage}[b]{0.32\columnwidth}\raggedright\strut
\textbf{Coste amortizado}\strut
\end{minipage}\tabularnewline
\midrule
\endhead
\begin{minipage}[t]{0.29\columnwidth}\raggedright\strut
Dispositivo móvil\strut
\end{minipage} & \begin{minipage}[t]{0.18\columnwidth}\raggedright\strut
300\euro{}\strut
\end{minipage} & \begin{minipage}[t]{0.32\columnwidth}\raggedright\strut
25\euro{}\strut
\end{minipage}\tabularnewline
\begin{minipage}[t]{0.29\columnwidth}\raggedright\strut
Ordenador portátil\strut
\end{minipage} & \begin{minipage}[t]{0.18\columnwidth}\raggedright\strut
800\euro{}\strut
\end{minipage} & \begin{minipage}[t]{0.32\columnwidth}\raggedright\strut
66,67\euro{}\strut
\end{minipage}\tabularnewline
\midrule
\begin{minipage}[t]{0.29\columnwidth}\raggedright\strut
\textbf{Total}\strut
\end{minipage} & \begin{minipage}[t]{0.18\columnwidth}\raggedright\strut
1.100\euro{}\strut
\end{minipage} & \begin{minipage}[t]{0.32\columnwidth}\raggedright\strut
91,67\euro{}\strut
\end{minipage}\tabularnewline
\bottomrule
\caption{Costes de \emph{hardware}.}
\end{longtable}
\newpage
\textbf{Costes de \emph{software}:}

En este apartado se revisan todos los costes en licencias de
\emph{software} no gratuito. Se considera que la amortización del
\emph{software} ronda los 2 años.

\begin{longtable}[]{@{}lrr@{}}
\toprule
\begin{minipage}[b]{0.24\columnwidth}\raggedright\strut
\textbf{Concepto}\strut
\end{minipage} & \begin{minipage}[b]{0.18\columnwidth}\raggedright\strut
\textbf{Coste}\strut
\end{minipage} & \begin{minipage}[b]{0.32\columnwidth}\raggedright\strut
\textbf{Coste amortizado}\strut
\end{minipage}\tabularnewline
\midrule
\endhead
\begin{minipage}[t]{0.24\columnwidth}\raggedright\strut
Windows 10 Pro\strut
\end{minipage} & \begin{minipage}[t]{0.18\columnwidth}\raggedright\strut
279\euro{}\strut
\end{minipage} & \begin{minipage}[t]{0.32\columnwidth}\raggedright\strut
58,13\euro{}\strut
\end{minipage}\tabularnewline
\begin{minipage}[t]{0.24\columnwidth}\raggedright\strut
Creately\strut
\end{minipage} & \begin{minipage}[t]{0.18\columnwidth}\raggedright\strut
5\euro{}\strut
\end{minipage} & \begin{minipage}[t]{0.32\columnwidth}\raggedright\strut
1,04\euro{}\strut
\end{minipage}\tabularnewline
\midrule
\begin{minipage}[t]{0.24\columnwidth}\raggedright\strut
\textbf{Total}\strut
\end{minipage} & \begin{minipage}[t]{0.18\columnwidth}\raggedright\strut
284\euro{}\strut
\end{minipage} & \begin{minipage}[t]{0.32\columnwidth}\raggedright\strut
59,17\euro{}\strut
\end{minipage}\tabularnewline
\bottomrule
\caption{Costes de \emph{software}.}
\end{longtable}

\textbf{Costes varios:}

En este apartado se revisan el resto de costes del proyecto.

\begin{longtable}[]{@{}lr@{}}
\toprule
\begin{minipage}[b]{0.48\columnwidth}\raggedright\strut
\textbf{Concepto}\strut
\end{minipage} & \begin{minipage}[b]{0.18\columnwidth}\raggedright\strut
\textbf{Coste}\strut
\end{minipage}\tabularnewline
\midrule
\endhead
\begin{minipage}[t]{0.48\columnwidth}\raggedright\strut
Dominio gobees.io\strut
\end{minipage} & \begin{minipage}[t]{0.18\columnwidth}\raggedright\strut
31,90\euro{}\strut
\end{minipage}\tabularnewline
\begin{minipage}[t]{0.48\columnwidth}\raggedright\strut
Cuenta Google Play\strut
\end{minipage} & \begin{minipage}[t]{0.18\columnwidth}\raggedright\strut
25\euro{}\strut
\end{minipage}\tabularnewline
\begin{minipage}[t]{0.48\columnwidth}\raggedright\strut
Memoria impresa y cartel\strut
\end{minipage} & \begin{minipage}[t]{0.18\columnwidth}\raggedright\strut
50\euro{}\strut
\end{minipage}\tabularnewline
\begin{minipage}[t]{0.48\columnwidth}\raggedright\strut
Alquiler de oficina\strut
\end{minipage} & \begin{minipage}[t]{0.18\columnwidth}\raggedright\strut
500\euro{}\strut
\end{minipage}\tabularnewline
\begin{minipage}[t]{0.48\columnwidth}\raggedright\strut
Internet\strut
\end{minipage} & \begin{minipage}[t]{0.18\columnwidth}\raggedright\strut
150\euro{}\strut
\end{minipage}\tabularnewline
\begin{minipage}[t]{0.48\columnwidth}\raggedright\strut
Material de apicultura de prueba\strut
\end{minipage} & \begin{minipage}[t]{0.18\columnwidth}\raggedright\strut
150\euro{}\strut
\end{minipage}\tabularnewline
\midrule
\begin{minipage}[t]{0.48\columnwidth}\raggedright\strut
\textbf{Total}\strut
\end{minipage} & \begin{minipage}[t]{0.18\columnwidth}\raggedright\strut
906,90\euro{}\strut
\end{minipage}\tabularnewline
\bottomrule
\caption{Costes varios.}
\end{longtable}

\textbf{Costes totales:}

El sumatorio de todos los costes es el siguiente:

\begin{longtable}[]{@{}lr@{}}
\toprule
\begin{minipage}[b]{0.22\columnwidth}\raggedright\strut
\textbf{Concepto}\strut
\end{minipage} & \begin{minipage}[b]{0.22\columnwidth}\raggedright\strut
\textbf{Coste}\strut
\end{minipage}\tabularnewline
\midrule
\endhead
\begin{minipage}[t]{0.22\columnwidth}\raggedright\strut
Personal\strut
\end{minipage} & \begin{minipage}[t]{0.22\columnwidth}\raggedright\strut
9.074,40\euro{}\strut
\end{minipage}\tabularnewline
\begin{minipage}[t]{0.22\columnwidth}\raggedright\strut
\emph{Hardware}\strut
\end{minipage} & \begin{minipage}[t]{0.22\columnwidth}\raggedright\strut
91,67\euro{}\strut
\end{minipage}\tabularnewline
\begin{minipage}[t]{0.22\columnwidth}\raggedright\strut
\emph{Software}\strut
\end{minipage} & \begin{minipage}[t]{0.22\columnwidth}\raggedright\strut
59,17\euro{}\strut
\end{minipage}\tabularnewline
\begin{minipage}[t]{0.22\columnwidth}\raggedright\strut
Varios\strut
\end{minipage} & \begin{minipage}[t]{0.22\columnwidth}\raggedright\strut
906,90\euro{}\strut
\end{minipage}\tabularnewline
\midrule
\begin{minipage}[t]{0.22\columnwidth}\raggedright\strut
Total\strut
\end{minipage} & \begin{minipage}[t]{0.22\columnwidth}\raggedright\strut
10.132,14\euro{}\strut
\end{minipage}\tabularnewline
\bottomrule
\caption{Costes totales.}
\end{longtable}

\subsubsection{Beneficios}\label{beneficios}

La aplicación desarrollada se distribuirá de forma gratuita y sin
publicidad, por lo que a corto plazo no se obtendrán beneficios.

La forma de monetizar la aplicación será en una segunda fase, cuando se
desarrolle una plataforma en la nube que sincronice la información de
varios dispositivos y permita el acceso remoto a la información.

Se considerarán tres tipos de suscripciones:

\begin{longtable}[]{@{}lllll@{}}
\toprule
\begin{minipage}[b]{0.16\columnwidth}\raggedright\strut
\textbf{Tipo}\strut
\end{minipage} & \begin{minipage}[b]{0.19\columnwidth}\raggedright\strut
\textbf{Colmenares}\strut
\end{minipage} & \begin{minipage}[b]{0.17\columnwidth}\raggedright\strut
\textbf{Colmenas}\strut
\end{minipage} & \begin{minipage}[b]{0.20\columnwidth}\raggedright\strut
\textbf{Plataformas}\strut
\end{minipage} & \begin{minipage}[b]{0.15\columnwidth}\raggedright\strut
\textbf{Precio}\strut
\end{minipage}\tabularnewline
\midrule
\endhead
\begin{minipage}[t]{0.16\columnwidth}\raggedright\strut
Hobby\strut
\end{minipage} & \begin{minipage}[t]{0.19\columnwidth}\raggedright\strut
1\strut
\end{minipage} & \begin{minipage}[t]{0.17\columnwidth}\raggedright\strut
10\strut
\end{minipage} & \begin{minipage}[t]{0.20\columnwidth}\raggedright\strut
App / Cloud\strut
\end{minipage} & \begin{minipage}[t]{0.15\columnwidth}\raggedright\strut
Gratis\strut
\end{minipage}\tabularnewline
\begin{minipage}[t]{0.16\columnwidth}\raggedright\strut
Amateur\strut
\end{minipage} & \begin{minipage}[t]{0.19\columnwidth}\raggedright\strut
5\strut
\end{minipage} & \begin{minipage}[t]{0.17\columnwidth}\raggedright\strut
100\strut
\end{minipage} & \begin{minipage}[t]{0.20\columnwidth}\raggedright\strut
App / Cloud\strut
\end{minipage} & \begin{minipage}[t]{0.15\columnwidth}\raggedright\strut
5\euro{}/mes\strut
\end{minipage}\tabularnewline
\begin{minipage}[t]{0.16\columnwidth}\raggedright\strut
Profesional\strut
\end{minipage} & \begin{minipage}[t]{0.19\columnwidth}\raggedright\strut
Ilimitados\strut
\end{minipage} & \begin{minipage}[t]{0.17\columnwidth}\raggedright\strut
Ilimitados\strut
\end{minipage} & \begin{minipage}[t]{0.20\columnwidth}\raggedright\strut
App / Cloud\strut
\end{minipage} & \begin{minipage}[t]{0.15\columnwidth}\raggedright\strut
20\euro{}/mes\strut
\end{minipage}\tabularnewline
\bottomrule
\caption{Tipos de suscripciones.}
\end{longtable}

\subsection{Viabilidad legal}\label{viabilidad-legal}

En esta sección se discutirán los temas relacionados con las licencias.
Tanto del propio \emph{software}, como de su documentación, imágenes y
vídeos.

``En Derecho, una licencia es un contrato mediante el cual una persona
recibe de otra el derecho de uso, de copia, de distribución, de estudio
y de modificación (en el caso del \emph{Software} Libre) de varios de
sus bienes, normalmente de carácter no tangible o intelectual, pudiendo
darse a cambio del pago de un monto determinado por el uso de los
mismos.'' \citep{wiki:licencia}

\subsubsection{Software}\label{software}

En primer lugar, vamos a analizar cuál sería la licencia más conveniente
para nuestro proyecto. Por un lado, somos nosotros los que podemos
elegir qué derechos queremos proporcionar a los usuarios y cuáles no.
Sin embargo, estamos limitados por los derechos que nos conceden a
nosotros las licencias de las dependencias utilizadas en el proyecto.

A continuación, se muestran las licencias de las dependencias usadas.

\begin{longtable}[]{@{}llll@{}}
\toprule
\begin{minipage}[b]{0.18\columnwidth}\raggedright\strut
Dependencia\strut
\end{minipage} & \begin{minipage}[b]{0.10\columnwidth}\raggedright\strut
Versión\strut
\end{minipage} & \begin{minipage}[b]{0.49\columnwidth}\raggedright\strut
Descripción\strut
\end{minipage} & \begin{minipage}[b]{0.11\columnwidth}\raggedright\strut
Licencia\strut
\end{minipage}\tabularnewline
\midrule
\endhead
\begin{minipage}[t]{0.18\columnwidth}\raggedright\strut
Android Support Library\strut
\end{minipage} & \begin{minipage}[t]{0.08\columnwidth}\raggedright\strut
25.1.0\strut
\end{minipage} & \begin{minipage}[t]{0.49\columnwidth}\raggedright\strut
Biblioteca de compatibilidad de Android.\strut
\end{minipage} & \begin{minipage}[t]{0.11\columnwidth}\raggedright\strut
Apache v2.0\strut
\end{minipage}\tabularnewline
\begin{minipage}[t]{0.18\columnwidth}\raggedright\strut
OpenCV\strut
\end{minipage} & \begin{minipage}[t]{0.08\columnwidth}\raggedright\strut
3.1.0\strut
\end{minipage} & \begin{minipage}[t]{0.49\columnwidth}\raggedright\strut
Biblioteca de visión artificial.\strut
\end{minipage} & \begin{minipage}[t]{0.11\columnwidth}\raggedright\strut
BSD\strut
\end{minipage}\tabularnewline
\begin{minipage}[t]{0.18\columnwidth}\raggedright\strut
Google Play Services\strut
\end{minipage} & \begin{minipage}[t]{0.08\columnwidth}\raggedright\strut
10.0.1\strut
\end{minipage} & \begin{minipage}[t]{0.49\columnwidth}\raggedright\strut
Biblioteca que proporciona acceso a diferentes servicios, entre ellos,
localización.\strut
\end{minipage} & \begin{minipage}[t]{0.11\columnwidth}\raggedright\strut
Apache v2.0\strut
\end{minipage}\tabularnewline
\begin{minipage}[t]{0.18\columnwidth}\raggedright\strut
Guava\strut
\end{minipage} & \begin{minipage}[t]{0.08\columnwidth}\raggedright\strut
20.0\strut
\end{minipage} & \begin{minipage}[t]{0.49\columnwidth}\raggedright\strut
Conjunto de bibliotecas comunes para Java.\strut
\end{minipage} & \begin{minipage}[t]{0.11\columnwidth}\raggedright\strut
Apache v2.0\strut
\end{minipage}\tabularnewline
\begin{minipage}[t]{0.18\columnwidth}\raggedright\strut
RoundedImage\strut
\end{minipage} & \begin{minipage}[t]{0.08\columnwidth}\raggedright\strut
2.3.0\strut
\end{minipage} & \begin{minipage}[t]{0.49\columnwidth}\raggedright\strut
Componente para mostrar imágenes redondeadas en Android.\strut
\end{minipage} & \begin{minipage}[t]{0.11\columnwidth}\raggedright\strut
Apache v2.0\strut
\end{minipage}\tabularnewline
\begin{minipage}[t]{0.18\columnwidth}\raggedright\strut
MPChart\strut
\end{minipage} & \begin{minipage}[t]{0.08\columnwidth}\raggedright\strut
3.0.1\strut
\end{minipage} & \begin{minipage}[t]{0.49\columnwidth}\raggedright\strut
Biblioteca de gráficos para Android.\strut
\end{minipage} & \begin{minipage}[t]{0.11\columnwidth}\raggedright\strut
Apache v2.0\strut
\end{minipage}\tabularnewline
\begin{minipage}[t]{0.18\columnwidth}\raggedright\strut
VNTPicker Preference\strut
\end{minipage} & \begin{minipage}[t]{0.08\columnwidth}\raggedright\strut
1.0.0\strut
\end{minipage} & \begin{minipage}[t]{0.49\columnwidth}\raggedright\strut
Componente para seleccionar valores numéricos.\strut
\end{minipage} & \begin{minipage}[t]{0.11\columnwidth}\raggedright\strut
Apache v2.0\strut
\end{minipage}\tabularnewline
\begin{minipage}[t]{0.18\columnwidth}\raggedright\strut
Permission Utils\strut
\end{minipage} & \begin{minipage}[t]{0.08\columnwidth}\raggedright\strut
1.0.6\strut
\end{minipage} & \begin{minipage}[t]{0.49\columnwidth}\raggedright\strut
Biblioteca que facilita la gestión de permisos en tiempo de
ejecución.\strut
\end{minipage} & \begin{minipage}[t]{0.11\columnwidth}\raggedright\strut
MIT\strut
\end{minipage}\tabularnewline
\begin{minipage}[t]{0.18\columnwidth}\raggedright\strut
JUnit\strut
\end{minipage} & \begin{minipage}[t]{0.08\columnwidth}\raggedright\strut
4.12\strut
\end{minipage} & \begin{minipage}[t]{0.49\columnwidth}\raggedright\strut
Framework para \emph{testing} unitario en Java.\strut
\end{minipage} & \begin{minipage}[t]{0.11\columnwidth}\raggedright\strut
EPL\strut
\end{minipage}\tabularnewline
\begin{minipage}[t]{0.18\columnwidth}\raggedright\strut
Mockito\strut
\end{minipage} & \begin{minipage}[t]{0.08\columnwidth}\raggedright\strut
2.0.2\strut
\end{minipage} & \begin{minipage}[t]{0.49\columnwidth}\raggedright\strut
Framework para \emph{mocking} en Java.\strut
\end{minipage} & \begin{minipage}[t]{0.11\columnwidth}\raggedright\strut
MIT\strut
\end{minipage}\tabularnewline
\begin{minipage}[t]{0.18\columnwidth}\raggedright\strut
SLF4J\strut
\end{minipage} & \begin{minipage}[t]{0.08\columnwidth}\raggedright\strut
1.7.21\strut
\end{minipage} & \begin{minipage}[t]{0.49\columnwidth}\raggedright\strut
API para \emph{logging} en Java.\strut
\end{minipage} & \begin{minipage}[t]{0.11\columnwidth}\raggedright\strut
MIT\strut
\end{minipage}\tabularnewline
\begin{minipage}[t]{0.18\columnwidth}\raggedright\strut
Apache Log4j\strut
\end{minipage} & \begin{minipage}[t]{0.08\columnwidth}\raggedright\strut
1.7.21\strut
\end{minipage} & \begin{minipage}[t]{0.49\columnwidth}\raggedright\strut
Biblioteca para \emph{logging} en Java.\strut
\end{minipage} & \begin{minipage}[t]{0.11\columnwidth}\raggedright\strut
Apache v2.0\strut
\end{minipage}\tabularnewline
\begin{minipage}[t]{0.18\columnwidth}\raggedright\strut
Android JSON\strut
\end{minipage} & \begin{minipage}[t]{0.08\columnwidth}\raggedright\strut
20160810\strut
\end{minipage} & \begin{minipage}[t]{0.49\columnwidth}\raggedright\strut
Biblioteca para trabajar con JSON.\strut
\end{minipage} & \begin{minipage}[t]{0.11\columnwidth}\raggedright\strut
Apache v2.0\strut
\end{minipage}\tabularnewline
\begin{minipage}[t]{0.18\columnwidth}\raggedright\strut
Espresso\strut
\end{minipage} & \begin{minipage}[t]{0.08\columnwidth}\raggedright\strut
2.2.2\strut
\end{minipage} & \begin{minipage}[t]{0.49\columnwidth}\raggedright\strut
Framework de \emph{testing} para Android.\strut
\end{minipage} & \begin{minipage}[t]{0.11\columnwidth}\raggedright\strut
Apache v2.0\strut
\end{minipage}\tabularnewline
\bottomrule
\caption{Dependencias del proyecto.}
\end{longtable}

Por lo tanto, tenemos que escoger una licencia para nuestro proyecto que
sea compatible con Apache v2.0, BSD, MIT y EPL. En el siguiente gráfico
mostramos la compatibilidad entre estas licencias, así como su grado de
permisividad.

\imagen{licenses_compatibility}{Compatibilidad entre licencias.}

Podemos observar que la licencia más restrictiva (en el sentido de
obligaciones a cumplir) es la \emph{Eclipse Public License} que posee la
librería JUnit.

La forma de monetización del proyecto se realizará mediante
suscripciones a una plataforma \emph{cloud} que permitirá la
sincronización entre varios dispositivos, entre otras funcionalidades.
Por lo tanto, la liberación del código del proyecto no pone en peligro
su monetización, sino todo lo contrario, abre la puerta a que la
comunidad \emph{Open Source} aporte valor adicional a nuestro proyecto.
El permitir la distribución de la app libremente y de forma gratuita
también nos es beneficioso, ya que aumenta las posibilidades de recibir
nuevas suscripciones de usuarios. Y por último, no nos importaría que
otras empresas se basaran en nuestro código fuente para desarrollar sus
productos, siempre los liberaran bajo una licencia de código abierto
para que nosotros también pudiéramos aprovechar las mejoras que hubieran
realizado.

Teniendo en cuenta todo lo anterior, la licencia que más se ajusta a
nuestras pretensiones es la \emph{GNU General Public License v3.0}, que,
de forma resumida, establece lo siguiente: \citep{license:gplv3}

\begin{longtable}[]{@{}lll@{}}
\toprule
\begin{minipage}[b]{0.19\columnwidth}\raggedright\strut
Derechos\strut
\end{minipage} & \begin{minipage}[b]{0.40\columnwidth}\raggedright\strut
Condiciones\strut
\end{minipage} & \begin{minipage}[b]{0.32\columnwidth}\raggedright\strut
Limitaciones\strut
\end{minipage}\tabularnewline
\midrule
\endhead
\begin{minipage}[t]{0.19\columnwidth}\raggedright\strut
Uso comercial.\strut
\end{minipage} & \begin{minipage}[t]{0.40\columnwidth}\raggedright\strut
Liberar código fuente.\strut
\end{minipage} & \begin{minipage}[t]{0.32\columnwidth}\raggedright\strut
Limitación de responsabilidad.\strut
\end{minipage}\tabularnewline
\begin{minipage}[t]{0.19\columnwidth}\raggedright\strut
Distribución.\strut
\end{minipage} & \begin{minipage}[t]{0.40\columnwidth}\raggedright\strut
Nota sobre la licencia y copyright.\strut
\end{minipage} & \begin{minipage}[t]{0.32\columnwidth}\raggedright\strut
Sin garantías.\strut
\end{minipage}\tabularnewline
\begin{minipage}[t]{0.19\columnwidth}\raggedright\strut
Modificación.\strut
\end{minipage} & \begin{minipage}[t]{0.40\columnwidth}\raggedright\strut
Modificaciones bajo la misma licencia.\strut
\end{minipage} & \begin{minipage}[t]{0.32\columnwidth}\raggedright\strut
\strut
\end{minipage}\tabularnewline
\begin{minipage}[t]{0.19\columnwidth}\raggedright\strut
Uso de patentes.\strut
\end{minipage} & \begin{minipage}[t]{0.40\columnwidth}\raggedright\strut
Indicar modificaciones realizadas.\strut
\end{minipage} & \begin{minipage}[t]{0.32\columnwidth}\raggedright\strut
\strut
\end{minipage}\tabularnewline
\begin{minipage}[t]{0.19\columnwidth}\raggedright\strut
Uso privado.\strut
\end{minipage} & \begin{minipage}[t]{0.40\columnwidth}\raggedright\strut
\strut
\end{minipage} & \begin{minipage}[t]{0.32\columnwidth}\raggedright\strut
\strut
\end{minipage}\tabularnewline
\bottomrule
\caption{Resumen de la licencia GLPv3.}
\end{longtable}

Sin embargo, GPL v3.0 no es compatible con la licencia EPL que posee
JUnit. Ya que, la EPL requiere que ``cualquier distribución del trabajo
conceda a todos los destinatarios una licencia para las patentes que
pudieran tener que cubrir las modificaciones que han hecho''
\citep{license:epl}. Esto supone que los destinatarios pueden añadir una
restricción adicional, hecho que prohíbe rotundamente GPL: ``{[}que el
distribuidor{]} no imponga ninguna restricción más sobre el ejercicio de
los derechos concedidos a los beneficiarios'' \citep{license:gplv3}.

Tras analizar otras licencias alternativas, no se ha encontrado ninguna
compatible con EPL y, a la vez, con nuestras pretensiones. Por lo que
finalmente se ha tomado la decisión de utilizar dos licencias para el
código fuente del proyecto. Por un lado, todo el código fuente de la
aplicación se ha licenciado bajo GPL v3.0. Mientras que el código fuente
de testeo, que hace uso de código licenciado bajo EPL (JUnit), se ha
liberado bajo licencia Apache v2.0, la cual sí que es compatible con
EPL.

\subsubsection{Documentación}\label{documentaciuxf3n}

Aunque se puede utilizar también la licencia GPL v3.0 para licenciar la
documentación, no es lo más recomendable. Ya que contiene numerosas
cláusulas que solo tienen sentido cuando se habla de código fuente. Por
ejemplo, si alguien quisiese distribuir una copia de la documentación de
forma impresa, estaría obligado a proporcionar también una copia del
código fuente.

Por lo que se ha decido utilizar una licencia \emph{Creative Commons},
las cuales están más enfocadas a licenciar este tipo de material. En
concreto, se ha elegido la \emph{Creative Commons Attribution 4.0
International} (CC-BY-4.0). Que establece lo siguiente:
\citep{license:ccby4}

\begin{longtable}[]{@{}lll@{}}
\toprule
\begin{minipage}[b]{0.17\columnwidth}\raggedright\strut
Derechos\strut
\end{minipage} & \begin{minipage}[b]{0.32\columnwidth}\raggedright\strut
Condiciones\strut
\end{minipage} & \begin{minipage}[b]{0.43\columnwidth}\raggedright\strut
Limitaciones\strut
\end{minipage}\tabularnewline
\midrule
\endhead
\begin{minipage}[t]{0.17\columnwidth}\raggedright\strut
Uso comercial.\strut
\end{minipage} & \begin{minipage}[t]{0.32\columnwidth}\raggedright\strut
Nota sobre la licencia y copyright.\strut
\end{minipage} & \begin{minipage}[t]{0.43\columnwidth}\raggedright\strut
Limitación de responsabilidad.\strut
\end{minipage}\tabularnewline
\begin{minipage}[t]{0.17\columnwidth}\raggedright\strut
Distribución.\strut
\end{minipage} & \begin{minipage}[t]{0.32\columnwidth}\raggedright\strut
Indicar modificaciones realizadas.\strut
\end{minipage} & \begin{minipage}[t]{0.43\columnwidth}\raggedright\strut
Sin garantías.\strut
\end{minipage}\tabularnewline
\begin{minipage}[t]{0.17\columnwidth}\raggedright\strut
Modificación.\strut
\end{minipage} & \begin{minipage}[t]{0.32\columnwidth}\raggedright\strut
\strut
\end{minipage} & \begin{minipage}[t]{0.43\columnwidth}\raggedright\strut
No proporciona derechos sobre marcas registradas.\strut
\end{minipage}\tabularnewline
\begin{minipage}[t]{0.17\columnwidth}\raggedright\strut
Uso privado.\strut
\end{minipage} & \begin{minipage}[t]{0.32\columnwidth}\raggedright\strut
\strut
\end{minipage} & \begin{minipage}[t]{0.43\columnwidth}\raggedright\strut
No proporciona derechos sobre patentes.\strut
\end{minipage}\tabularnewline
\bottomrule
\caption{Resumen de la licencia CC-BY-4.0.}
\end{longtable}

\subsubsection{Imágenes y vídeos}\label{imuxe1genes-y-vuxeddeos}

En la documentación no se ha utilizado ninguna imagen de terceros, todas
las imágenes son propias del proyecto y cuentan con la misma licencia
que la documentación (CC-BY-4.0).

El \emph{dataset} de vídeos de prueba también se encuentra bajo la misma
licencia.

Por otro lado, en la aplicación se han utilizado dos fuentes de imágenes
de terceros:

\begin{longtable}[]{@{}lll@{}}
\toprule
\begin{minipage}[b]{0.28\columnwidth}\raggedright\strut
Fuente\strut
\end{minipage} & \begin{minipage}[b]{0.46\columnwidth}\raggedright\strut
Descripción\strut
\end{minipage} & \begin{minipage}[b]{0.17\columnwidth}\raggedright\strut
Licencia\strut
\end{minipage}\tabularnewline
\midrule
\endhead
\begin{minipage}[t]{0.28\columnwidth}\raggedright\strut
Material design icons\strut
\end{minipage} & \begin{minipage}[t]{0.46\columnwidth}\raggedright\strut
Conjunto de iconos oficial de Google.\strut
\end{minipage} & \begin{minipage}[t]{0.17\columnwidth}\raggedright\strut
Apache v2.0\strut
\end{minipage}\tabularnewline
\begin{minipage}[t]{0.28\columnwidth}\raggedright\strut
Simple Weather Icons\strut
\end{minipage} & \begin{minipage}[t]{0.46\columnwidth}\raggedright\strut
Conjunto de iconos meteorológicos.\strut
\end{minipage} & \begin{minipage}[t]{0.17\columnwidth}\raggedright\strut
Apache v2.0\strut
\end{minipage}\tabularnewline
\bottomrule
\caption{Fuentes de imágenes de terceros.}
\end{longtable}

Aunque ambos autores renuncian a la obligación de especificar
explícitamente su autoría, se les ha mencionado en la sección
``Licencias de software libre'' de la aplicación.

El resto de imágenes y gráficos utilizados son de autoría propia y se
distribuyen también bajo CC-BY-4.0.3.

\subsubsection{Resumen}\label{resumen-1}

En la siguiente tabla se resumen las licencias que posee el proyecto.

\begin{longtable}[]{@{}ll@{}}
\toprule
\begin{minipage}[b]{0.31\columnwidth}\raggedright\strut
Recurso\strut
\end{minipage} & \begin{minipage}[b]{0.21\columnwidth}\raggedright\strut
Licencia\strut
\end{minipage}\tabularnewline
\midrule
\endhead
\begin{minipage}[t]{0.31\columnwidth}\raggedright\strut
Código fuente app\strut
\end{minipage} & \begin{minipage}[t]{0.21\columnwidth}\raggedright\strut
GPLv3\strut
\end{minipage}\tabularnewline
\begin{minipage}[t]{0.31\columnwidth}\raggedright\strut
Código fuente tests\strut
\end{minipage} & \begin{minipage}[t]{0.21\columnwidth}\raggedright\strut
Apache v2.0\strut
\end{minipage}\tabularnewline
\begin{minipage}[t]{0.31\columnwidth}\raggedright\strut
Documentación\strut
\end{minipage} & \begin{minipage}[t]{0.21\columnwidth}\raggedright\strut
CC-BY-4.0\strut
\end{minipage}\tabularnewline
\begin{minipage}[t]{0.31\columnwidth}\raggedright\strut
Imágenes\strut
\end{minipage} & \begin{minipage}[t]{0.21\columnwidth}\raggedright\strut
CC-BY-4.0\strut
\end{minipage}\tabularnewline
\begin{minipage}[t]{0.31\columnwidth}\raggedright\strut
Vídeos\strut
\end{minipage} & \begin{minipage}[t]{0.21\columnwidth}\raggedright\strut
CC-BY-4.0\strut
\end{minipage}\tabularnewline
\bottomrule
\caption{Resumen de las licencias del proyecto.}
\end{longtable}

\apendice{Especificación de Requisitos}

\section{Introducción}\label{introduccion-requisitos}

Este anexo recoge la especificación de requisitos que define el
comportamiento del sistema desarrollado. Posee un doble objetivo: servir
como documento contractual entre el cliente y el equipo de desarrollo y
como documentación correspondiente al análisis a la aplicación.

Se han seguido las recomendaciones del estándar IEEE
830-1998, que manifiesta que una buena especificación de requisitos
\emph{software} debe ser: \citep{ieee_830_1998}

\begin{itemize}
\tightlist
\item
  \textbf{Completa}: todos los requerimientos deben estar reflejados en
  ella y todas las referencias deben estar definidas.
\item
  \textbf{Consistente}: debe ser coherente con los propios
  requerimientos y también con otros documentos de especificación.
\item
  \textbf{Inequívoca}: la redacción debe ser clara de modo que no se
  pueda mal interpretar.
\item
  \textbf{Correcta}: el software debe cumplir con los requisitos de la
  especificación.
\item
  \textbf{Trazable}\emph{: s}e refiere a la posibilidad de verificar la
  historia, ubicación o aplicación de un ítem a través de su
  identificación almacenada y documentada.
\item
  \textbf{Priorizable}: los requisitos deben poder organizarse
  jerárquicamente según su relevancia para el negocio y clasificándolos
  en esenciales, condicionales y opcionales.
\item
  \textbf{Modificable}: aunque todo requerimiento es modificable, se
  refiere a que debe ser fácilmente modificable.
\item
  \textbf{Verificable}: debe existir un método finito sin costo para
  poder probarlo.
\end{itemize}

\section{Objetivos generales}\label{objetivos-generales}

El proyecto persigue los siguientes objetivos generales:

\begin{itemize}
\tightlist
\item
  Desarrollar una aplicación para \emph{smartphones} que permita la
  monitorización de la actividad de vuelo de una colmena a través de su
  cámara.
\item
  Facilitar la interpretación de los datos recogidos mediante
  representaciones gráficas.
\item
  Aportar información extra a los datos de actividad que ayude en la
  toma de decisiones.
\item
  Almacenar todos los datos generados de forma estructurada y fácilmente
  accesible.
\end{itemize}

\section{Catálogo de requisitos}\label{catalogo-de-requisitos}

A continuación, se enumeran los requisitos específicos derivados de los
objetivos generales del proyecto.

\subsection{Requisitos funcionales}\label{requisitos-funcionales}

\begin{itemize}
\tightlist
\item
  \textbf{RF-1 Gestión de colmenares:} la aplicación tiene que ser capaz
  de gestionar colmenares.

  \begin{itemize}
  \tightlist
  \item
    \textbf{RF-1.1 Añadir colmenar:} el usuario debe poder añadir un
    nuevo colmenar con un nombre, una localización y unas notas
    específicas.

    \begin{itemize}
    \tightlist
    \item
      \textbf{RF-1.1.1: Obtener localización:} la aplicación tiene que
      ser capaz de obtener la localización actual del usuario.
    \end{itemize}
  \item
    \textbf{RF-1.2 Editar colmenar:} el usuario debe poder editar la
    información de un colmenar ya existente.

    \begin{itemize}
    \tightlist
    \item
      \textbf{RF-1.2.1: Obtener localización:} la aplicación tiene que
      ser capaz de obtener la localización actual del usuario.
    \end{itemize}
  \item
    \textbf{RF-1.3 Eliminar colmenar:} el usuario debe poder eliminar un
    colmenar ya existente junto con toda su información asociada.
  \item
    \textbf{RF-1.4 Listar colmenares:} el usuario debe poder listar
    todos los colmenares existentes.

    \begin{itemize}
    \tightlist
    \item
      \textbf{RF-1.4.1 Obtención de información meteorológica:} la
      aplicación tiene que ser capaz de obtener la información
      meteorológica de cada uno de los colmenares.
    \end{itemize}
  \item
    \textbf{RF-1.5 Ver colmenar:} el usuario debe poder visualizar toda
    la información relativa a un determinado colmenar.

    \begin{itemize}
    \tightlist
    \item
      \textbf{RF-1.5.1 Obtención de información meteorológica:} la
      aplicación tiene que ser capaz de obtener la información
      meteorológica relativa a un determinado colmenar.
    \end{itemize}
  \end{itemize}
\item
  \textbf{RF-2 Gestión de colmenas:} la aplicación tiene que ser capaz
  de gestionar colmenas.

  \begin{itemize}
  \tightlist
  \item
    \textbf{RF-2.1 Añadir colmena:} el usuario debe poder añadir una
    nueva colmena con un nombre y unas notas específicas.
  \item
    \textbf{RF-2.2 Editar colmena:} el usuario debe poder editar la
    información de una colmena ya existente.
  \item
    \textbf{RF-2.3 Eliminar colmena:} el usuario debe poder eliminar una
    colmena ya existente junto con toda su información asociada.
  \item
    \textbf{RF-2.4 Listar colmenas:} el usuario debe poder listar todas
    las colmenas existentes en un determinado colmenar.
  \item
    \textbf{RF-2.5 Ver colmena:} el usuario debe poder visualizar toda
    la información relativa a una determinada colmena.
  \end{itemize}
\item
  \textbf{RF-3 Gestión de grabaciones:} la aplicación tiene que ser
  capaz de gestionar grabaciones.

  \begin{itemize}
  \tightlist
  \item
    \textbf{RF-3.1 Añadir grabación:} la aplicación tiene que ser capaz
    de crear una nueva grabación a partir de los datos de
    monitorización.
  \item
    \textbf{RF-3.2 Eliminar grabación:} el usuario debe poder eliminar
    una grabación ya existente junto con toda su información asociada.
  \item
    \textbf{RF-3.3 Listar grabaciones:} el usuario debe poder listar
    todas las grabaciones existentes de una determinada colmena.
  \item
    \textbf{RF-3.4 Ver grabación:} el usuario debe poder visualizar toda
    la información relativa a una determinada grabación.
  \end{itemize}
\item
  \textbf{RF-4 Monitorización de la actividad de vuelo:} el usuario
  tiene que ser capaz de monitorizar la actividad de vuelo de una
  colmena a partir de una determinada parametrización de esta.

  \begin{itemize}
  \tightlist
  \item
    \textbf{RF-4.1 Previsualización:} el usuario debe poder
    previsualizar la salida del algoritmo de conteo de abejas.
  \item
    \textbf{RF-4.2 Configurar monitorización:} el usuario debe poder
    configurar todos los parámetros relativos a la monitorización.
  \item
    \textbf{RF-4.3 Obtención de información meteorológica:} la
    aplicación tiene que ser capaz de obtener la información
    meteorológica relativa a un determinado colmenar.
  \end{itemize}
\item
  \textbf{RF-5 Configuración de la aplicación:} el usuario debe poder
  configurar todos los parámetros disponibles en la aplicación, como el
  idioma o las unidades meteorológicas.
\item
  \textbf{RF-6 Ayuda de la aplicación:} el usuario debe poder obtener
  ayuda sobre cada una de las funcionalidades de la aplicación.
\item
  \textbf{RF-7 Información de la aplicación:} el usuario debe poder
  obtener información sobre la aplicación, compartirla o enviar
  sugerencias.
\end{itemize}

\subsection{Requisitos no funcionales}\label{requisitos-no-funcionales}

\begin{itemize}
\tightlist
\item
  \textbf{RNF-1 Usabilidad:} la aplicación debe ser intuitiva, con una
  curva baja de aprendizaje, errores explicativos y adaptada al entorno
  de trabajo.
\item
  \textbf{RNF-2 Rendimiento:} la aplicación tiene que tener unos tiempos
  de carga y procesado aceptables en un dispositivo móvil de gama media.
  La pantalla nunca deberá quedar congelada.
\item
  \textbf{RNF-3 Capacidad y Escalabilidad:} la aplicación tiene que
  estar preparada para una recogida de datos continuada y debe permitir
  la adición de nuevas funcionalidades de forma sencilla.
\item
  \textbf{RNF-4 Disponibilidad:} la aplicación debe estar siempre
  disponible para su uso, independientemente de la localización, la no
  disponibilidad de internet, o cualquier otro factor.
\item
  \textbf{RNF-5 Seguridad:} la aplicación debe gestionar de forma
  adecuada todos los datos de carácter sensible, como claves,
  \emph{tokens}, etc.
\item
  \textbf{RNF-6 Mantenibilidad}: la aplicación debe ser desarrollada de
  acuerdo a algún patrón arquitectónico estándar que asegure
  escalabilidad, portabilidad, testabilidad, etc. Además, tiene que
  cumplir los estándares de código de Android.
\item
  \textbf{RNF-7 Soporte}: la aplicación debe dar soporte a versiones
  mayores o iguales a Android 4.4 (\emph{KitKat}).
\item
  \textbf{RNF-8 Monitorización}: la aplicación debe monitorizar
  correctamente la actividad de vuelo de una colmena cuando el
  dispositivo se coloca en posición cenital a la colmena, sobre un
  soporte estático y con un fondo claro y uniforme.
\item
  \textbf{RNF-9 Internacionalización}: la aplicación deberá estar
  preparada para soportar varios idiomas, localizando textos, unidades
  de medida, imágenes, etc.
\end{itemize}

\section{Especificación de
requisitos}\label{especificacion-de-requisitos-1}

En esta sección se mostrará el diagrama de casos de uso resultante y se
desarrollará cada uno de ellos.

\begin{landscape}
\subsection{Diagrama de casos de uso}\label{diagrama-de-casos-de-uso}
\imagenAncho{use_cases_diagram}{Diagrama de casos de uso.}{1.35}
\end{landscape}

\subsection{Actores}\label{actores}

Solo interactuará con el sistema un actor, que se corresponderá con la
figura del apicultor.

\subsection{Casos de uso}\label{casos-de-uso}

\begin{longtable}[H]{@{}ll@{}}
\toprule
\begin{minipage}[b]{0.23\columnwidth}\raggedright\strut
\textbf{CU-01}\strut
\end{minipage} & \begin{minipage}[b]{0.71\columnwidth}\raggedright\strut
\textbf{Gestión de colmenares}\strut
\end{minipage}\tabularnewline
\midrule
\endhead
\begin{minipage}[t]{0.23\columnwidth}\raggedright\strut
\textbf{Versión}\strut
\end{minipage} & \begin{minipage}[t]{0.71\columnwidth}\raggedright\strut
1.0\strut
\end{minipage}\tabularnewline
\begin{minipage}[t]{0.23\columnwidth}\raggedright\strut
\textbf{Autor}\strut
\end{minipage} & \begin{minipage}[t]{0.71\columnwidth}\raggedright\strut
David Miguel Lozano\strut
\end{minipage}\tabularnewline
\begin{minipage}[t]{0.23\columnwidth}\raggedright\strut
\textbf{Requisitos asociados}\strut
\end{minipage} & \begin{minipage}[t]{0.71\columnwidth}\raggedright\strut
RF-1, RF-1.1, RF-1.1.1, RF-1.2, RF-1.2.1, RF-1.3, RF-1.4, RF-1.5,
RF-1.5.1\strut
\end{minipage}\tabularnewline
\begin{minipage}[t]{0.23\columnwidth}\raggedright\strut
\textbf{Descripción}\strut
\end{minipage} & \begin{minipage}[t]{0.71\columnwidth}\raggedright\strut
Permite al usuario gestionar sus colmenares.\strut
\end{minipage}\tabularnewline
\begin{minipage}[t]{0.23\columnwidth}\raggedright\strut
\textbf{Precondición}\strut
\end{minipage} & \begin{minipage}[t]{0.71\columnwidth}\raggedright\strut
La base de datos se encuentra disponible.\strut
\end{minipage}\tabularnewline
\begin{minipage}[t]{0.23\columnwidth}\raggedright\strut
\textbf{Acciones}\strut
\end{minipage} & \begin{minipage}[t]{0.71\columnwidth}\raggedright\strut
\begin{enumerate}
\def\labelenumi{\arabic{enumi}.}
\tightlist
\item
  El usuario entra en la aplicación.
\item
  Se listan todos los colmenares.
\item
  Por cada colmenar se da la opción de ver detalle, editar o eliminar.
\item
  Se muestra un botón para añadir un colmenar.
\end{enumerate}\strut
\end{minipage}\tabularnewline
\begin{minipage}[t]{0.23\columnwidth}\raggedright\strut
\textbf{Postcondición}\strut
\end{minipage} & \begin{minipage}[t]{0.71\columnwidth}\raggedright\strut
El número de colmenares listado es igual al número de colmenares en la
base de datos.\strut
\end{minipage}\tabularnewline
\begin{minipage}[t]{0.23\columnwidth}\raggedright\strut
\textbf{Excepciones}\strut
\end{minipage} & \begin{minipage}[t]{0.71\columnwidth}\raggedright\strut
\begin{itemize}
\tightlist
\item
  Error al cargar colmenares (mensaje).
\item
  No existe ningún colmenar (vista especial).
\end{itemize}\strut
\end{minipage}\tabularnewline
\begin{minipage}[t]{0.23\columnwidth}\raggedright\strut
\textbf{Importancia}\strut
\end{minipage} & \begin{minipage}[t]{0.71\columnwidth}\raggedright\strut
Alta\strut
\end{minipage}\tabularnewline
\bottomrule
\caption{CU-01 Gestión de colmenares.}
\end{longtable}

\begin{longtable}[H]{@{}ll@{}}
\toprule
\begin{minipage}[b]{0.24\columnwidth}\raggedright\strut
\textbf{CU-02}\strut
\end{minipage} & \begin{minipage}[b]{0.71\columnwidth}\raggedright\strut
\textbf{Añadir colmenar}\strut
\end{minipage}\tabularnewline
\midrule
\endhead
\begin{minipage}[t]{0.24\columnwidth}\raggedright\strut
\textbf{Versión}\strut
\end{minipage} & \begin{minipage}[t]{0.71\columnwidth}\raggedright\strut
1.0\strut
\end{minipage}\tabularnewline
\begin{minipage}[t]{0.24\columnwidth}\raggedright\strut
\textbf{Autor}\strut
\end{minipage} & \begin{minipage}[t]{0.71\columnwidth}\raggedright\strut
David Miguel Lozano\strut
\end{minipage}\tabularnewline
\begin{minipage}[t]{0.24\columnwidth}\raggedright\strut
\textbf{Requisitos asociados}\strut
\end{minipage} & \begin{minipage}[t]{0.71\columnwidth}\raggedright\strut
RF-1.1, RF-1.1.1\strut
\end{minipage}\tabularnewline
\begin{minipage}[t]{0.24\columnwidth}\raggedright\strut
\textbf{Descripción}\strut
\end{minipage} & \begin{minipage}[t]{0.71\columnwidth}\raggedright\strut
Permite al usuario añadir un nuevo colmenar.\strut
\end{minipage}\tabularnewline
\begin{minipage}[t]{0.24\columnwidth}\raggedright\strut
\textbf{Precondición}\strut
\end{minipage} & \begin{minipage}[t]{0.71\columnwidth}\raggedright\strut
La base de datos se encuentra disponible.\strut
\end{minipage}\tabularnewline
\begin{minipage}[t]{0.24\columnwidth}\raggedright\strut
\textbf{Acciones}\strut
\end{minipage} & \begin{minipage}[t]{0.71\columnwidth}\raggedright\strut
\begin{enumerate}
\def\labelenumi{\arabic{enumi}.}
\tightlist
\item
  El usuario presiona en el botón de añadir colmenar.
\item
  Se muestra el formulario para introducir los datos del colmenar.
\item
  El usuario introduce el nombre.
\item
  El usuario pulsa obtener localización (opcional).

  \begin{enumerate}
  \def\labelenumii{\alph{enumii}.}
  \tightlist
  \item
    Se obtiene la localización del usuario.
  \end{enumerate}
\item
  El usuario introduce notas sobre el colmenar (opcional).
\item
  El usuario pulsa el botón de aceptar.
\item
  Si no hay ningún error, se guarda un nuevo colmenar con los datos
  introducidos.
\item
  Volver a Gestión de colmenares.
\end{enumerate}\strut
\end{minipage}\tabularnewline
\begin{minipage}[t]{0.24\columnwidth}\raggedright\strut
\textbf{Postcondición}\strut
\end{minipage} & \begin{minipage}[t]{0.71\columnwidth}\raggedright\strut
Existe un colmenar más en la base de datos.\strut
\end{minipage}\tabularnewline
\begin{minipage}[t]{0.24\columnwidth}\raggedright\strut
\textbf{Excepciones}\strut
\end{minipage} & \begin{minipage}[t]{0.71\columnwidth}\raggedright\strut
\begin{itemize}
\tightlist
\item
  Error al guardar colmenar (mensaje).
\item
  No se ha introducido nombre del colmenar (resaltar).
\end{itemize}\strut
\end{minipage}\tabularnewline
\begin{minipage}[t]{0.24\columnwidth}\raggedright\strut
\textbf{Importancia}\strut
\end{minipage} & \begin{minipage}[t]{0.71\columnwidth}\raggedright\strut
Alta\strut
\end{minipage}\tabularnewline
\bottomrule
\caption{CU-02 Añadir colmenar.}
\end{longtable}

\begin{longtable}[H]{@{}ll@{}}
\toprule
\begin{minipage}[b]{0.26\columnwidth}\raggedright\strut
\textbf{CU-03}\strut
\end{minipage} & \begin{minipage}[b]{0.68\columnwidth}\raggedright\strut
\textbf{Editar colmenar}\strut
\end{minipage}\tabularnewline
\midrule
\endhead
\begin{minipage}[t]{0.26\columnwidth}\raggedright\strut
\textbf{Versión}\strut
\end{minipage} & \begin{minipage}[t]{0.68\columnwidth}\raggedright\strut
1.0\strut
\end{minipage}\tabularnewline
\begin{minipage}[t]{0.26\columnwidth}\raggedright\strut
\textbf{Autor}\strut
\end{minipage} & \begin{minipage}[t]{0.68\columnwidth}\raggedright\strut
David Miguel Lozano\strut
\end{minipage}\tabularnewline
\begin{minipage}[t]{0.26\columnwidth}\raggedright\strut
\textbf{Requisitos asociados}\strut
\end{minipage} & \begin{minipage}[t]{0.68\columnwidth}\raggedright\strut
RF-1.2, RF-1.2.1\strut
\end{minipage}\tabularnewline
\begin{minipage}[t]{0.26\columnwidth}\raggedright\strut
\textbf{Descripción}\strut
\end{minipage} & \begin{minipage}[t]{0.68\columnwidth}\raggedright\strut
Permite al usuario editar un colmenar ya existente.\strut
\end{minipage}\tabularnewline
\begin{minipage}[t]{0.26\columnwidth}\raggedright\strut
\textbf{Precondición}\strut
\end{minipage} & \begin{minipage}[t]{0.68\columnwidth}\raggedright\strut
La base de datos se encuentra disponible.

El colmenar a editar existe.\strut
\end{minipage}\tabularnewline
\begin{minipage}[t]{0.26\columnwidth}\raggedright\strut
\textbf{Acciones}\strut
\end{minipage} & \begin{minipage}[t]{0.68\columnwidth}\raggedright\strut
\begin{enumerate}
\def\labelenumi{\arabic{enumi}.}
\tightlist
\item
  El usuario selecciona un colmenar para editar.
\item
  Se obtienen los datos del colmenar de la base de datos.
\item
  Se rellena el formulario de edición con los datos del colmenar.
\item
  El usuario edita alguno de los campos.
\item
  Si el usuario pulsa obtener localización.

  \begin{enumerate}
  \def\labelenumii{\alph{enumii}.}
  \tightlist
  \item
    Se obtiene la localización del usuario.
  \end{enumerate}
\item
  El usuario pulsa el botón aceptar.
\item
  Si no hay ningún error, se actualiza el colmenar en la base de datos.
\end{enumerate}\strut
\end{minipage}\tabularnewline
\begin{minipage}[t]{0.26\columnwidth}\raggedright\strut
\textbf{Postcondición}\strut
\end{minipage} & \begin{minipage}[t]{0.68\columnwidth}\raggedright\strut
La información del colmenar en la base de datos ha sido
actualizada.\strut
\end{minipage}\tabularnewline
\begin{minipage}[t]{0.26\columnwidth}\raggedright\strut
\textbf{Excepciones}\strut
\end{minipage} & \begin{minipage}[t]{0.68\columnwidth}\raggedright\strut
\begin{itemize}
\tightlist
\item
  Error al guardar colmenar (mensaje).
\item
  No se ha introducido nombre del colmenar (resaltar).
\end{itemize}\strut
\end{minipage}\tabularnewline
\begin{minipage}[t]{0.26\columnwidth}\raggedright\strut
\textbf{Importancia}\strut
\end{minipage} & \begin{minipage}[t]{0.68\columnwidth}\raggedright\strut
Alta\strut
\end{minipage}\tabularnewline
\bottomrule
\caption{CU-03 Editar colmenar.}
\end{longtable}

\begin{longtable}[H]{@{}ll@{}}
\toprule
\begin{minipage}[b]{0.29\columnwidth}\raggedright\strut
\textbf{CU-04}\strut
\end{minipage} & \begin{minipage}[b]{0.65\columnwidth}\raggedright\strut
\textbf{Eliminar colmenar}\strut
\end{minipage}\tabularnewline
\midrule
\endhead
\begin{minipage}[t]{0.29\columnwidth}\raggedright\strut
\textbf{Versión}\strut
\end{minipage} & \begin{minipage}[t]{0.65\columnwidth}\raggedright\strut
1.0\strut
\end{minipage}\tabularnewline
\begin{minipage}[t]{0.29\columnwidth}\raggedright\strut
\textbf{Autor}\strut
\end{minipage} & \begin{minipage}[t]{0.65\columnwidth}\raggedright\strut
David Miguel Lozano\strut
\end{minipage}\tabularnewline
\begin{minipage}[t]{0.29\columnwidth}\raggedright\strut
\textbf{Requisitos asociados}\strut
\end{minipage} & \begin{minipage}[t]{0.65\columnwidth}\raggedright\strut
RF-1.3\strut
\end{minipage}\tabularnewline
\begin{minipage}[t]{0.29\columnwidth}\raggedright\strut
\textbf{Descripción}\strut
\end{minipage} & \begin{minipage}[t]{0.65\columnwidth}\raggedright\strut
Permite al usuario eliminar un colmenar ya existente.\strut
\end{minipage}\tabularnewline
\begin{minipage}[t]{0.29\columnwidth}\raggedright\strut
\textbf{Precondición}\strut
\end{minipage} & \begin{minipage}[t]{0.65\columnwidth}\raggedright\strut
La base de datos se encuentra disponible.

El colmenar a eliminar existe.\strut
\end{minipage}\tabularnewline
\begin{minipage}[t]{0.29\columnwidth}\raggedright\strut
\textbf{Acciones}\strut
\end{minipage} & \begin{minipage}[t]{0.65\columnwidth}\raggedright\strut
\begin{enumerate}
\def\labelenumi{\arabic{enumi}.}
\tightlist
\item
  El usuario selecciona un colmenar para eliminar.
\item
  Se eliminan los datos de ese colmenar de la base de datos.
\item
  Se elimina el colmenar de la vista.
\item
  Se informa al usuario.
\end{enumerate}\strut
\end{minipage}\tabularnewline
\begin{minipage}[t]{0.29\columnwidth}\raggedright\strut
\textbf{Postcondición}\strut
\end{minipage} & \begin{minipage}[t]{0.65\columnwidth}\raggedright\strut
Existe un colmenar menos en la base de datos.\strut
\end{minipage}\tabularnewline
\begin{minipage}[t]{0.29\columnwidth}\raggedright\strut
\textbf{Excepciones}\strut
\end{minipage} & \begin{minipage}[t]{0.65\columnwidth}\raggedright\strut
\begin{itemize}
\tightlist
\item
  Error al eliminar colmenar (mensaje).
\end{itemize}\strut
\end{minipage}\tabularnewline
\begin{minipage}[t]{0.29\columnwidth}\raggedright\strut
\textbf{Importancia}\strut
\end{minipage} & \begin{minipage}[t]{0.65\columnwidth}\raggedright\strut
Alta\strut
\end{minipage}\tabularnewline
\bottomrule
\caption{CU-04 Eliminar colmenar.}
\end{longtable}

\begin{longtable}[H]{@{}ll@{}}
\toprule
\begin{minipage}[b]{0.26\columnwidth}\raggedright\strut
\textbf{CU-05}\strut
\end{minipage} & \begin{minipage}[b]{0.68\columnwidth}\raggedright\strut
\textbf{Listar colmenares}\strut
\end{minipage}\tabularnewline
\midrule
\endhead
\begin{minipage}[t]{0.26\columnwidth}\raggedright\strut
\textbf{Versión}\strut
\end{minipage} & \begin{minipage}[t]{0.68\columnwidth}\raggedright\strut
1.0\strut
\end{minipage}\tabularnewline
\begin{minipage}[t]{0.26\columnwidth}\raggedright\strut
\textbf{Autor}\strut
\end{minipage} & \begin{minipage}[t]{0.68\columnwidth}\raggedright\strut
David Miguel Lozano\strut
\end{minipage}\tabularnewline
\begin{minipage}[t]{0.26\columnwidth}\raggedright\strut
\textbf{Requisitos asociados}\strut
\end{minipage} & \begin{minipage}[t]{0.68\columnwidth}\raggedright\strut
RF-1.4, RF-1.4.1\strut
\end{minipage}\tabularnewline
\begin{minipage}[t]{0.26\columnwidth}\raggedright\strut
\textbf{Descripción}\strut
\end{minipage} & \begin{minipage}[t]{0.68\columnwidth}\raggedright\strut
Permite al usuario listar todos sus colmenares. Por cada colmenar se
muestra el nombre, el número de colmenas y la condición meteorológica y
temperatura actuales.\strut
\end{minipage}\tabularnewline
\begin{minipage}[t]{0.26\columnwidth}\raggedright\strut
\textbf{Precondición}\strut
\end{minipage} & \begin{minipage}[t]{0.68\columnwidth}\raggedright\strut
La base de datos se encuentra disponible.\strut
\end{minipage}\tabularnewline
\begin{minipage}[t]{0.26\columnwidth}\raggedright\strut
\textbf{Acciones}\strut
\end{minipage} & \begin{minipage}[t]{0.68\columnwidth}\raggedright\strut
\begin{enumerate}
\def\labelenumi{\arabic{enumi}.}
\tightlist
\item
  El usuario accede a Gestionar Colmenares.
\item
  Se obtienen todos los colmenares de la base de datos.
\item
  Se actualiza su información meteorológica si no se dispone de esta o
  la que se dispone es de hace más de 15 minutos.
\item
  Se muestran la lista de colmenares. Cada elemento de la lista posee el
  nombre del colmenar, el número de colmenas y la condición
  meteorológica y temperatura de ese colmenar.
\end{enumerate}\strut
\end{minipage}\tabularnewline
\begin{minipage}[t]{0.26\columnwidth}\raggedright\strut
\textbf{Postcondición}\strut
\end{minipage} & \begin{minipage}[t]{0.68\columnwidth}\raggedright\strut
-\strut
\end{minipage}\tabularnewline
\begin{minipage}[t]{0.26\columnwidth}\raggedright\strut
\textbf{Excepciones}\strut
\end{minipage} & \begin{minipage}[t]{0.68\columnwidth}\raggedright\strut
\begin{itemize}
\tightlist
\item
  Error al cargar colmenares (mensaje).
\item
  No existen colmenares (vista especial).
\item
  No existe conexión a internet (mensaje).
\item
  Error al recuperar la información meteorológica (mensaje).
\end{itemize}\strut
\end{minipage}\tabularnewline
\begin{minipage}[t]{0.26\columnwidth}\raggedright\strut
\textbf{Importancia}\strut
\end{minipage} & \begin{minipage}[t]{0.68\columnwidth}\raggedright\strut
Alta\strut
\end{minipage}\tabularnewline
\bottomrule
\caption{CU-05 Listar colmenares.}
\end{longtable}

\begin{longtable}[H]{@{}ll@{}}
\toprule
\begin{minipage}[b]{0.268\columnwidth}\raggedright\strut
\textbf{CU-06}\strut
\end{minipage} & \begin{minipage}[b]{0.76\columnwidth}\raggedright\strut
\textbf{Ver colmenar}\strut
\end{minipage}\tabularnewline
\midrule
\endhead
\begin{minipage}[t]{0.268\columnwidth}\raggedright\strut
\textbf{Versión}\strut
\end{minipage} & \begin{minipage}[t]{0.76\columnwidth}\raggedright\strut
1.0\strut
\end{minipage}\tabularnewline
\begin{minipage}[t]{0.268\columnwidth}\raggedright\strut
\textbf{Autor}\strut
\end{minipage} & \begin{minipage}[t]{0.76\columnwidth}\raggedright\strut
David Miguel Lozano\strut
\end{minipage}\tabularnewline
\begin{minipage}[t]{0.268\columnwidth}\raggedright\strut
\textbf{Requisitos asociados}\strut
\end{minipage} & \begin{minipage}[t]{0.76\columnwidth}\raggedright\strut
RF-1.5, RF-1.5.1\strut
\end{minipage}\tabularnewline
\begin{minipage}[t]{0.268\columnwidth}\raggedright\strut
\textbf{Descripción}\strut
\end{minipage} & \begin{minipage}[t]{0.76\columnwidth}\raggedright\strut
Permite al usuario visualizar toda la información relativa a un
determinado colmenar existente.\strut
\end{minipage}\tabularnewline
\begin{minipage}[t]{0.268\columnwidth}\raggedright\strut
\textbf{Precondición}\strut
\end{minipage} & \begin{minipage}[t]{0.76\columnwidth}\raggedright\strut
La base de datos se encuentra disponible.

El colmenar a visualizar existe.\strut
\end{minipage}\tabularnewline
\begin{minipage}[t]{0.268\columnwidth}\raggedright\strut
\textbf{Acciones}\strut
\end{minipage} & \begin{minipage}[t]{0.76\columnwidth}\raggedright\strut
\begin{enumerate}
\def\labelenumi{\arabic{enumi}.}
\tightlist
\item
  El usuario selecciona un colmenar para visualizar.
\item
  Se obtienen los datos del colmenar de la base de datos (incluidas sus
  colmenas).
\item
  Se actualiza su información meteorológica si no se dispone de esta o
  la que se dispone es de hace más de 15 minutos.
\item
  Se muestra una lista con sus colmenas.
\item
  Se muestra la información general del colmenar (localización, número
  de colmenas, última revisión y notas).
\item
  Se muestra la información meteorológica en detalle.
\end{enumerate}\strut
\end{minipage}\tabularnewline
\begin{minipage}[t]{0.268\columnwidth}\raggedright\strut
\textbf{Postcondición}\strut
\end{minipage} & \begin{minipage}[t]{0.76\columnwidth}\raggedright\strut
-\strut
\end{minipage}\tabularnewline
\begin{minipage}[t]{0.268\columnwidth}\raggedright\strut
\textbf{Excepciones}\strut
\end{minipage} & \begin{minipage}[t]{0.76\columnwidth}\raggedright\strut
\begin{itemize}
\tightlist
\item
  Error al cargar colmenar (mensaje).
\item
  No existe conexión a internet (mensaje).
\item
  Error al recuperar la información meteorológica (mensaje).
\end{itemize}\strut
\end{minipage}\tabularnewline
\begin{minipage}[t]{0.268\columnwidth}\raggedright\strut
\textbf{Importancia}\strut
\end{minipage} & \begin{minipage}[t]{0.76\columnwidth}\raggedright\strut
Alta\strut
\end{minipage}\tabularnewline
\bottomrule
\caption{CU-06 Ver colmenar.}
\end{longtable}

\begin{longtable}[H]{@{}ll@{}}
\toprule
\begin{minipage}[b]{0.26\columnwidth}\raggedright\strut
\textbf{CU-07}\strut
\end{minipage} & \begin{minipage}[b]{0.68\columnwidth}\raggedright\strut
\textbf{Obtener localización}\strut
\end{minipage}\tabularnewline
\midrule
\endhead
\begin{minipage}[t]{0.26\columnwidth}\raggedright\strut
\textbf{Versión}\strut
\end{minipage} & \begin{minipage}[t]{0.68\columnwidth}\raggedright\strut
1.0\strut
\end{minipage}\tabularnewline
\begin{minipage}[t]{0.26\columnwidth}\raggedright\strut
\textbf{Autor}\strut
\end{minipage} & \begin{minipage}[t]{0.68\columnwidth}\raggedright\strut
David Miguel Lozano\strut
\end{minipage}\tabularnewline
\begin{minipage}[t]{0.26\columnwidth}\raggedright\strut
\textbf{Requisitos asociados}\strut
\end{minipage} & \begin{minipage}[t]{0.68\columnwidth}\raggedright\strut
RF-1.1.1, RF-1.2.1\strut
\end{minipage}\tabularnewline
\begin{minipage}[t]{0.26\columnwidth}\raggedright\strut
\textbf{Descripción}\strut
\end{minipage} & \begin{minipage}[t]{0.68\columnwidth}\raggedright\strut
Permite obtener la localización actual del usuario.\strut
\end{minipage}\tabularnewline
\begin{minipage}[t]{0.26\columnwidth}\raggedright\strut
\textbf{Precondición}\strut
\end{minipage} & \begin{minipage}[t]{0.68\columnwidth}\raggedright\strut
Se poseen permisos de acceso a la localización.\strut
\end{minipage}\tabularnewline
\begin{minipage}[t]{0.26\columnwidth}\raggedright\strut
\textbf{Acciones}\strut
\end{minipage} & \begin{minipage}[t]{0.68\columnwidth}\raggedright\strut
\begin{enumerate}
\def\labelenumi{\arabic{enumi}.}
\tightlist
\item
  El usuario selecciona obtener localización actual.
\item
  La aplicación se conecta al servicio de localización.
\item
  El servicio de localización va devolviendo ubicaciones, cada vez más
  precisas.
\item
  Cuando el usuario considera la localización suficientemente buena,
  vuelve a presionar el botón de localización para detener la
  localización. Si no lo hace, se detendrá automáticamente al cambiar de
  actividad.
\item
  Se devuelve la localización obtenida.
\end{enumerate}\strut
\end{minipage}\tabularnewline
\begin{minipage}[t]{0.26\columnwidth}\raggedright\strut
\textbf{Postcondición}\strut
\end{minipage} & \begin{minipage}[t]{0.68\columnwidth}\raggedright\strut
Las coordenadas devueltas son válidas.\strut
\end{minipage}\tabularnewline
\begin{minipage}[t]{0.26\columnwidth}\raggedright\strut
\textbf{Excepciones}\strut
\end{minipage} & \begin{minipage}[t]{0.68\columnwidth}\raggedright\strut
\begin{itemize}
\tightlist
\item
  No se poseen permisos de localización (solicitar).
\item
  Error de conexión con el GPS (mensaje).
\end{itemize}\strut
\end{minipage}\tabularnewline
\begin{minipage}[t]{0.26\columnwidth}\raggedright\strut
\textbf{Importancia}\strut
\end{minipage} & \begin{minipage}[t]{0.68\columnwidth}\raggedright\strut
Alta\strut
\end{minipage}\tabularnewline
\bottomrule
\caption{CU-07 Obtener localización.}
\end{longtable}

\begin{longtable}[H]{@{}ll@{}}
\toprule
\begin{minipage}[b]{0.20\columnwidth}\raggedright\strut
\textbf{CU-08}\strut
\end{minipage} & \begin{minipage}[b]{0.74\columnwidth}\raggedright\strut
\textbf{Obtener información meteorológica}\strut
\end{minipage}\tabularnewline
\midrule
\endhead
\begin{minipage}[t]{0.20\columnwidth}\raggedright\strut
\textbf{Versión}\strut
\end{minipage} & \begin{minipage}[t]{0.74\columnwidth}\raggedright\strut
1.0\strut
\end{minipage}\tabularnewline
\begin{minipage}[t]{0.20\columnwidth}\raggedright\strut
\textbf{Autor}\strut
\end{minipage} & \begin{minipage}[t]{0.74\columnwidth}\raggedright\strut
David Miguel Lozano\strut
\end{minipage}\tabularnewline
\begin{minipage}[t]{0.20\columnwidth}\raggedright\strut
\textbf{Requisitos asociados}\strut
\end{minipage} & \begin{minipage}[t]{0.74\columnwidth}\raggedright\strut
RF-1.4.1, RF-1.5.1\strut
\end{minipage}\tabularnewline
\begin{minipage}[t]{0.20\columnwidth}\raggedright\strut
\textbf{Descripción}\strut
\end{minipage} & \begin{minipage}[t]{0.74\columnwidth}\raggedright\strut
Permite obtener la información meteorológica actual en un determinado
colmenar.\strut
\end{minipage}\tabularnewline
\begin{minipage}[t]{0.20\columnwidth}\raggedright\strut
\textbf{Precondición}\strut
\end{minipage} & \begin{minipage}[t]{0.74\columnwidth}\raggedright\strut
Se poseen permisos de acceso a internet.

El colmenar existe y posee localización.\strut
\end{minipage}\tabularnewline
\begin{minipage}[t]{0.20\columnwidth}\raggedright\strut
\textbf{Acciones}\strut
\end{minipage} & \begin{minipage}[t]{0.74\columnwidth}\raggedright\strut
\begin{enumerate}
\def\labelenumi{\arabic{enumi}.}
\tightlist
\item
  El sistema ejecuta la orden de actualizar información meteorológica
  para un determinado colmenar.
\item
  Se obtiene la ubicación del colmenar de la base de datos.
\item
  Se realiza una consulta a la API de \emph{OpenWeatherMap}.
\item
  Se procesan los datos recibidos.
\item
  Se devuelven los datos recibidos.
\end{enumerate}\strut
\end{minipage}\tabularnewline
\begin{minipage}[t]{0.20\columnwidth}\raggedright\strut
\textbf{Postcondición}\strut
\end{minipage} & \begin{minipage}[t]{0.74\columnwidth}\raggedright\strut
La información meteorológica devuelta es válida.\strut
\end{minipage}\tabularnewline
\begin{minipage}[t]{0.20\columnwidth}\raggedright\strut
\textbf{Excepciones}\strut
\end{minipage} & \begin{minipage}[t]{0.74\columnwidth}\raggedright\strut
\begin{itemize}
\tightlist
\item
  No se poseen permisos de internet (solicitar).
\item
  El colmenar no tiene localización (ignorar petición).
\end{itemize}\strut
\end{minipage}\tabularnewline
\begin{minipage}[t]{0.20\columnwidth}\raggedright\strut
\textbf{Importancia}\strut
\end{minipage} & \begin{minipage}[t]{0.74\columnwidth}\raggedright\strut
Alta\strut
\end{minipage}\tabularnewline
\bottomrule
\caption{CU-08 Obtener información meteorológica.}
\end{longtable}

\begin{longtable}[H]{@{}ll@{}}
\toprule
\begin{minipage}[b]{0.21\columnwidth}\raggedright\strut
\textbf{CU-09}\strut
\end{minipage} & \begin{minipage}[b]{0.73\columnwidth}\raggedright\strut
\textbf{Gestión de colmenas}\strut
\end{minipage}\tabularnewline
\midrule
\endhead
\begin{minipage}[t]{0.21\columnwidth}\raggedright\strut
\textbf{Versión}\strut
\end{minipage} & \begin{minipage}[t]{0.73\columnwidth}\raggedright\strut
1.0\strut
\end{minipage}\tabularnewline
\begin{minipage}[t]{0.21\columnwidth}\raggedright\strut
\textbf{Autor}\strut
\end{minipage} & \begin{minipage}[t]{0.73\columnwidth}\raggedright\strut
David Miguel Lozano\strut
\end{minipage}\tabularnewline
\begin{minipage}[t]{0.21\columnwidth}\raggedright\strut
\textbf{Requisitos asociados}\strut
\end{minipage} & \begin{minipage}[t]{0.73\columnwidth}\raggedright\strut
RF-2, RF-2.1, RF-2.2, RF-2.3, RF-2.4, RF-2.5\strut
\end{minipage}\tabularnewline
\begin{minipage}[t]{0.21\columnwidth}\raggedright\strut
\textbf{Descripción}\strut
\end{minipage} & \begin{minipage}[t]{0.73\columnwidth}\raggedright\strut
Permite al usuario gestionar las colmenas de un determinado
colmenar.\strut
\end{minipage}\tabularnewline
\begin{minipage}[t]{0.21\columnwidth}\raggedright\strut
\textbf{Precondición}\strut
\end{minipage} & \begin{minipage}[t]{0.73\columnwidth}\raggedright\strut
La base de datos se encuentra disponible.

El colmenar existe.\strut
\end{minipage}\tabularnewline
\begin{minipage}[t]{0.21\columnwidth}\raggedright\strut
\textbf{Acciones}\strut
\end{minipage} & \begin{minipage}[t]{0.73\columnwidth}\raggedright\strut
\begin{enumerate}
\def\labelenumi{\arabic{enumi}.}
\tightlist
\item
  El usuario entra en la vista detalle de un colmenar.
\item
  Se listan todas las colmenas.
\item
  Por cada colmena se da la opción de ver detalle, editar o eliminar.
\item
  Se muestra un botón para añadir una colmena.
\end{enumerate}\strut
\end{minipage}\tabularnewline
\begin{minipage}[t]{0.21\columnwidth}\raggedright\strut
\textbf{Postcondición}\strut
\end{minipage} & \begin{minipage}[t]{0.73\columnwidth}\raggedright\strut
El número de colmenas listado es igual al número de colmenas de ese
colmenar en la base de datos.\strut
\end{minipage}\tabularnewline
\begin{minipage}[t]{0.21\columnwidth}\raggedright\strut
\textbf{Excepciones}\strut
\end{minipage} & \begin{minipage}[t]{0.73\columnwidth}\raggedright\strut
\begin{itemize}
\tightlist
\item
  Error al cargar colmenas (mensaje).
\item
  No existe ninguna colmena (vista especial).
\end{itemize}\strut
\end{minipage}\tabularnewline
\begin{minipage}[t]{0.21\columnwidth}\raggedright\strut
\textbf{Importancia}\strut
\end{minipage} & \begin{minipage}[t]{0.73\columnwidth}\raggedright\strut
Alta\strut
\end{minipage}\tabularnewline
\bottomrule
\caption{CU-09 Gestión de colmenas.}
\end{longtable}

\begin{longtable}[H]{@{}ll@{}}
\toprule
\begin{minipage}[b]{0.269\columnwidth}\raggedright\strut
\textbf{CU-10}\strut
\end{minipage} & \begin{minipage}[b]{0.75\columnwidth}\raggedright\strut
\textbf{Añadir colmena}\strut
\end{minipage}\tabularnewline
\midrule
\endhead
\begin{minipage}[t]{0.269\columnwidth}\raggedright\strut
\textbf{Versión}\strut
\end{minipage} & \begin{minipage}[t]{0.75\columnwidth}\raggedright\strut
1.0\strut
\end{minipage}\tabularnewline
\begin{minipage}[t]{0.269\columnwidth}\raggedright\strut
\textbf{Autor}\strut
\end{minipage} & \begin{minipage}[t]{0.75\columnwidth}\raggedright\strut
David Miguel Lozano\strut
\end{minipage}\tabularnewline
\begin{minipage}[t]{0.269\columnwidth}\raggedright\strut
\textbf{Requisitos asociados}\strut
\end{minipage} & \begin{minipage}[t]{0.75\columnwidth}\raggedright\strut
RF-2.1\strut
\end{minipage}\tabularnewline
\begin{minipage}[t]{0.269\columnwidth}\raggedright\strut
\textbf{Descripción}\strut
\end{minipage} & \begin{minipage}[t]{0.75\columnwidth}\raggedright\strut
Permite al usuario añadir una nueva colmena.\strut
\end{minipage}\tabularnewline
\begin{minipage}[t]{0.269\columnwidth}\raggedright\strut
\textbf{Precondición}\strut
\end{minipage} & \begin{minipage}[t]{0.75\columnwidth}\raggedright\strut
La base de datos se encuentra disponible.

El colmenar existe.\strut
\end{minipage}\tabularnewline
\begin{minipage}[t]{0.269\columnwidth}\raggedright\strut
\textbf{Acciones}\strut
\end{minipage} & \begin{minipage}[t]{0.75\columnwidth}\raggedright\strut
\begin{enumerate}
\def\labelenumi{\arabic{enumi}.}
\tightlist
\item
  El usuario presiona en el botón de añadir colmena.
\item
  Se muestra el formulario para introducir los datos de la colmena.
\item
  El usuario introduce el nombre.
\item
  El usuario introduce notas sobre el colmenar (opcional).
\item
  El usuario pulsa el botón de aceptar.
\item
  Si no hay ningún error, se guarda una nueva colmena con los datos
  introducidos y se asocia al colmenar.
\item
  Volver a Gestión de colmenas.
\end{enumerate}\strut
\end{minipage}\tabularnewline
\begin{minipage}[t]{0.269\columnwidth}\raggedright\strut
\textbf{Postcondición}\strut
\end{minipage} & \begin{minipage}[t]{0.75\columnwidth}\raggedright\strut
Existe una colmena más para ese colmenar en la base de datos.\strut
\end{minipage}\tabularnewline
\begin{minipage}[t]{0.269\columnwidth}\raggedright\strut
\textbf{Excepciones}\strut
\end{minipage} & \begin{minipage}[t]{0.75\columnwidth}\raggedright\strut
\begin{itemize}
\tightlist
\item
  Error al guardar colmena (mensaje).
\item
  No se ha introducido nombre de la colmena (resaltar).
\end{itemize}\strut
\end{minipage}\tabularnewline
\begin{minipage}[t]{0.269\columnwidth}\raggedright\strut
\textbf{Importancia}\strut
\end{minipage} & \begin{minipage}[t]{0.75\columnwidth}\raggedright\strut
Alta\strut
\end{minipage}\tabularnewline
\bottomrule
\caption{CU-10 Añadir colmena.}
\end{longtable}

\begin{longtable}[H]{@{}ll@{}}
\toprule
\begin{minipage}[b]{0.26\columnwidth}\raggedright\strut
\textbf{CU-11}\strut
\end{minipage} & \begin{minipage}[b]{0.68\columnwidth}\raggedright\strut
\textbf{Editar colmena}\strut
\end{minipage}\tabularnewline
\midrule
\endhead
\begin{minipage}[t]{0.26\columnwidth}\raggedright\strut
\textbf{Versión}\strut
\end{minipage} & \begin{minipage}[t]{0.68\columnwidth}\raggedright\strut
1.0\strut
\end{minipage}\tabularnewline
\begin{minipage}[t]{0.26\columnwidth}\raggedright\strut
\textbf{Autor}\strut
\end{minipage} & \begin{minipage}[t]{0.68\columnwidth}\raggedright\strut
David Miguel Lozano\strut
\end{minipage}\tabularnewline
\begin{minipage}[t]{0.26\columnwidth}\raggedright\strut
\textbf{Requisitos asociados}\strut
\end{minipage} & \begin{minipage}[t]{0.68\columnwidth}\raggedright\strut
RF-2.2\strut
\end{minipage}\tabularnewline
\begin{minipage}[t]{0.26\columnwidth}\raggedright\strut
\textbf{Descripción}\strut
\end{minipage} & \begin{minipage}[t]{0.68\columnwidth}\raggedright\strut
Permite al usuario editar una colmena ya existente.\strut
\end{minipage}\tabularnewline
\begin{minipage}[t]{0.26\columnwidth}\raggedright\strut
\textbf{Precondición}\strut
\end{minipage} & \begin{minipage}[t]{0.68\columnwidth}\raggedright\strut
La base de datos se encuentra disponible.

El colmenar existe.\strut
\end{minipage}\tabularnewline
\begin{minipage}[t]{0.26\columnwidth}\raggedright\strut
\textbf{Acciones}\strut
\end{minipage} & \begin{minipage}[t]{0.68\columnwidth}\raggedright\strut
\begin{enumerate}
\def\labelenumi{\arabic{enumi}.}
\tightlist
\item
  El usuario selecciona una colmena para editar.
\item
  Se obtienen los datos de la colmena de la base de datos.
\item
  Se rellena el formulario de edición con los datos del colmenar.
\item
  El usuario edita alguno de los campos.
\item
  El usuario pulsa el botón aceptar.
\item
  Si no hay ningún error, se actualiza la colmena en la base de datos.
\end{enumerate}\strut
\end{minipage}\tabularnewline
\begin{minipage}[t]{0.26\columnwidth}\raggedright\strut
\textbf{Postcondición}\strut
\end{minipage} & \begin{minipage}[t]{0.68\columnwidth}\raggedright\strut
La información de la colmena en la base de datos ha sido
actualizada.\strut
\end{minipage}\tabularnewline
\begin{minipage}[t]{0.26\columnwidth}\raggedright\strut
\textbf{Excepciones}\strut
\end{minipage} & \begin{minipage}[t]{0.68\columnwidth}\raggedright\strut
\begin{itemize}
\tightlist
\item
  Error al guardar colmena (mensaje).
\item
  No se ha introducido nombre de la colmena (resaltar).
\end{itemize}\strut
\end{minipage}\tabularnewline
\begin{minipage}[t]{0.26\columnwidth}\raggedright\strut
\textbf{Importancia}\strut
\end{minipage} & \begin{minipage}[t]{0.68\columnwidth}\raggedright\strut
Alta\strut
\end{minipage}\tabularnewline
\bottomrule
\caption{CU-11 Editar colmena.}
\end{longtable}

\begin{longtable}[]{@{}ll@{}}
\toprule
\begin{minipage}[b]{0.29\columnwidth}\raggedright\strut
\textbf{CU-12}\strut
\end{minipage} & \begin{minipage}[b]{0.65\columnwidth}\raggedright\strut
\textbf{Eliminar colmena}\strut
\end{minipage}\tabularnewline
\midrule
\endhead
\begin{minipage}[t]{0.29\columnwidth}\raggedright\strut
\textbf{Versión}\strut
\end{minipage} & \begin{minipage}[t]{0.65\columnwidth}\raggedright\strut
1.0\strut
\end{minipage}\tabularnewline
\begin{minipage}[t]{0.29\columnwidth}\raggedright\strut
\textbf{Autor}\strut
\end{minipage} & \begin{minipage}[t]{0.65\columnwidth}\raggedright\strut
David Miguel Lozano\strut
\end{minipage}\tabularnewline
\begin{minipage}[t]{0.29\columnwidth}\raggedright\strut
\textbf{Requisitos asociados}\strut
\end{minipage} & \begin{minipage}[t]{0.65\columnwidth}\raggedright\strut
RF-2.3\strut
\end{minipage}\tabularnewline
\begin{minipage}[t]{0.29\columnwidth}\raggedright\strut
\textbf{Descripción}\strut
\end{minipage} & \begin{minipage}[t]{0.65\columnwidth}\raggedright\strut
Permite al usuario eliminar una colmena ya existente.\strut
\end{minipage}\tabularnewline
\begin{minipage}[t]{0.29\columnwidth}\raggedright\strut
\textbf{Precondición}\strut
\end{minipage} & \begin{minipage}[t]{0.65\columnwidth}\raggedright\strut
La base de datos se encuentra disponible.

El colmenar existe.

La colmena a eliminar existe.\strut
\end{minipage}\tabularnewline
\begin{minipage}[t]{0.29\columnwidth}\raggedright\strut
\textbf{Acciones}\strut
\end{minipage} & \begin{minipage}[t]{0.65\columnwidth}\raggedright\strut
\begin{enumerate}
\def\labelenumi{\arabic{enumi}.}
\tightlist
\item
  El usuario selecciona una colmena para eliminar.
\item
  Se eliminan los datos de esa colmena de la base de datos.
\item
  Se elimina la colmena de la vista.
\item
  Se informa al usuario.
\end{enumerate}\strut
\end{minipage}\tabularnewline
\begin{minipage}[t]{0.29\columnwidth}\raggedright\strut
\textbf{Postcondición}\strut
\end{minipage} & \begin{minipage}[t]{0.65\columnwidth}\raggedright\strut
Existe una colmena menos en ese colmenar en la base de datos.\strut
\end{minipage}\tabularnewline
\begin{minipage}[t]{0.29\columnwidth}\raggedright\strut
\textbf{Excepciones}\strut
\end{minipage} & \begin{minipage}[t]{0.65\columnwidth}\raggedright\strut
\begin{itemize}
\tightlist
\item
  Error al eliminar colmena (mensaje).
\end{itemize}\strut
\end{minipage}\tabularnewline
\begin{minipage}[t]{0.29\columnwidth}\raggedright\strut
\textbf{Importancia}\strut
\end{minipage} & \begin{minipage}[t]{0.65\columnwidth}\raggedright\strut
Alta\strut
\end{minipage}\tabularnewline
\bottomrule
\caption{CU-12 Eliminar colmena.}
\end{longtable}

\begin{longtable}[H]{@{}ll@{}}
\toprule
\begin{minipage}[b]{0.26\columnwidth}\raggedright\strut
\textbf{CU-13}\strut
\end{minipage} & \begin{minipage}[b]{0.68\columnwidth}\raggedright\strut
\textbf{Listar colmenas}\strut
\end{minipage}\tabularnewline
\midrule
\endhead
\begin{minipage}[t]{0.26\columnwidth}\raggedright\strut
\textbf{Versión}\strut
\end{minipage} & \begin{minipage}[t]{0.68\columnwidth}\raggedright\strut
1.0\strut
\end{minipage}\tabularnewline
\begin{minipage}[t]{0.26\columnwidth}\raggedright\strut
\textbf{Autor}\strut
\end{minipage} & \begin{minipage}[t]{0.68\columnwidth}\raggedright\strut
David Miguel Lozano\strut
\end{minipage}\tabularnewline
\begin{minipage}[t]{0.26\columnwidth}\raggedright\strut
\textbf{Requisitos asociados}\strut
\end{minipage} & \begin{minipage}[t]{0.68\columnwidth}\raggedright\strut
RF-2.4\strut
\end{minipage}\tabularnewline
\begin{minipage}[t]{0.26\columnwidth}\raggedright\strut
\textbf{Descripción}\strut
\end{minipage} & \begin{minipage}[t]{0.68\columnwidth}\raggedright\strut
Permite al usuario listar todas las colmenas de un determinado colmenar.
Por cada colmena se muestra el nombre y la fecha de la última
revisión.\strut
\end{minipage}\tabularnewline
\begin{minipage}[t]{0.26\columnwidth}\raggedright\strut
\textbf{Precondición}\strut
\end{minipage} & \begin{minipage}[t]{0.68\columnwidth}\raggedright\strut
La base de datos se encuentra disponible.

El colmenar existe.\strut
\end{minipage}\tabularnewline
\begin{minipage}[t]{0.26\columnwidth}\raggedright\strut
\textbf{Acciones}\strut
\end{minipage} & \begin{minipage}[t]{0.68\columnwidth}\raggedright\strut
\begin{enumerate}
\def\labelenumi{\arabic{enumi}.}
\tightlist
\item
  El usuario accede a Gestionar Colmenas de un determinado colmenar.
\item
  Se obtienen todas las colmenas de ese colmenar de la base de datos.
\item
  Se muestran la lista de colmenas. Cada elemento de la lista posee el
  nombre de la colmena y la fecha de la última revisión.
\end{enumerate}\strut
\end{minipage}\tabularnewline
\begin{minipage}[t]{0.26\columnwidth}\raggedright\strut
\textbf{Postcondición}\strut
\end{minipage} & \begin{minipage}[t]{0.68\columnwidth}\raggedright\strut
-\strut
\end{minipage}\tabularnewline
\begin{minipage}[t]{0.26\columnwidth}\raggedright\strut
\textbf{Excepciones}\strut
\end{minipage} & \begin{minipage}[t]{0.68\columnwidth}\raggedright\strut
\begin{itemize}
\tightlist
\item
  Error al cargar colmenas (mensaje).
\item
  No existen colmenas (vista especial).
\end{itemize}\strut
\end{minipage}\tabularnewline
\begin{minipage}[t]{0.26\columnwidth}\raggedright\strut
\textbf{Importancia}\strut
\end{minipage} & \begin{minipage}[t]{0.68\columnwidth}\raggedright\strut
Alta\strut
\end{minipage}\tabularnewline
\bottomrule
\caption{CU-13 Listar colmenas.}
\end{longtable}

\begin{longtable}[H]{@{}ll@{}}
\toprule
\begin{minipage}[b]{0.21\columnwidth}\raggedright\strut
\textbf{CU-14}\strut
\end{minipage} & \begin{minipage}[b]{0.73\columnwidth}\raggedright\strut
\textbf{Ver colmena}\strut
\end{minipage}\tabularnewline
\midrule
\endhead
\begin{minipage}[t]{0.21\columnwidth}\raggedright\strut
\textbf{Versión}\strut
\end{minipage} & \begin{minipage}[t]{0.73\columnwidth}\raggedright\strut
1.0\strut
\end{minipage}\tabularnewline
\begin{minipage}[t]{0.21\columnwidth}\raggedright\strut
\textbf{Autor}\strut
\end{minipage} & \begin{minipage}[t]{0.73\columnwidth}\raggedright\strut
David Miguel Lozano\strut
\end{minipage}\tabularnewline
\begin{minipage}[t]{0.21\columnwidth}\raggedright\strut
\textbf{Requisitos asociados}\strut
\end{minipage} & \begin{minipage}[t]{0.73\columnwidth}\raggedright\strut
RF-2.5\strut
\end{minipage}\tabularnewline
\begin{minipage}[t]{0.21\columnwidth}\raggedright\strut
\textbf{Descripción}\strut
\end{minipage} & \begin{minipage}[t]{0.73\columnwidth}\raggedright\strut
Permite al usuario visualizar toda la información relativa a una
determinada colmena existente.\strut
\end{minipage}\tabularnewline
\begin{minipage}[t]{0.21\columnwidth}\raggedright\strut
\textbf{Precondición}\strut
\end{minipage} & \begin{minipage}[t]{0.73\columnwidth}\raggedright\strut
La base de datos se encuentra disponible.

El colmenar existe.

La colmena a visualizar existe.\strut
\end{minipage}\tabularnewline
\begin{minipage}[t]{0.21\columnwidth}\raggedright\strut
\textbf{Acciones}\strut
\end{minipage} & \begin{minipage}[t]{0.73\columnwidth}\raggedright\strut
\begin{enumerate}
\def\labelenumi{\arabic{enumi}.}
\tightlist
\item
  El usuario selecciona una colmena de un determinado colmenar para
  visualizar.
\item
  Se obtienen los datos de la colmena de la base de datos (incluidas sus
  grabaciones).
\item
  Se muestra una lista con sus grabaciones.
\item
  Se muestra la información general de la colmena (última revisión y
  notas).
\end{enumerate}\strut
\end{minipage}\tabularnewline
\begin{minipage}[t]{0.21\columnwidth}\raggedright\strut
\textbf{Postcondición}\strut
\end{minipage} & \begin{minipage}[t]{0.73\columnwidth}\raggedright\strut
-\strut
\end{minipage}\tabularnewline
\begin{minipage}[t]{0.21\columnwidth}\raggedright\strut
\textbf{Excepciones}\strut
\end{minipage} & \begin{minipage}[t]{0.73\columnwidth}\raggedright\strut
\begin{itemize}
\tightlist
\item
  Error al cargar colmena (mensaje).
\end{itemize}\strut
\end{minipage}\tabularnewline
\begin{minipage}[t]{0.21\columnwidth}\raggedright\strut
\textbf{Importancia}\strut
\end{minipage} & \begin{minipage}[t]{0.73\columnwidth}\raggedright\strut
Alta\strut
\end{minipage}\tabularnewline
\bottomrule
\caption{CU-14 Ver colmena.}
\end{longtable}

\begin{longtable}[H]{@{}ll@{}}
\toprule
\begin{minipage}[b]{0.20\columnwidth}\raggedright\strut
\textbf{CU-15}\strut
\end{minipage} & \begin{minipage}[b]{0.74\columnwidth}\raggedright\strut
\textbf{Gestión de grabaciones}\strut
\end{minipage}\tabularnewline
\midrule
\endhead
\begin{minipage}[t]{0.20\columnwidth}\raggedright\strut
\textbf{Versión}\strut
\end{minipage} & \begin{minipage}[t]{0.74\columnwidth}\raggedright\strut
1.0\strut
\end{minipage}\tabularnewline
\begin{minipage}[t]{0.20\columnwidth}\raggedright\strut
\textbf{Autor}\strut
\end{minipage} & \begin{minipage}[t]{0.74\columnwidth}\raggedright\strut
David Miguel Lozano\strut
\end{minipage}\tabularnewline
\begin{minipage}[t]{0.20\columnwidth}\raggedright\strut
\textbf{Requisitos asociados}\strut
\end{minipage} & \begin{minipage}[t]{0.74\columnwidth}\raggedright\strut
RF-3, RF-3.1, RF-3.2, RF-3.3, RF-3.4\strut
\end{minipage}\tabularnewline
\begin{minipage}[t]{0.20\columnwidth}\raggedright\strut
\textbf{Descripción}\strut
\end{minipage} & \begin{minipage}[t]{0.74\columnwidth}\raggedright\strut
Permite al usuario gestionar las grabaciones de una determinada
colmena.\strut
\end{minipage}\tabularnewline
\begin{minipage}[t]{0.20\columnwidth}\raggedright\strut
\textbf{Precondición}\strut
\end{minipage} & \begin{minipage}[t]{0.74\columnwidth}\raggedright\strut
La base de datos se encuentra disponible.

El colmenar y la colmena existen.\strut
\end{minipage}\tabularnewline
\begin{minipage}[t]{0.20\columnwidth}\raggedright\strut
\textbf{Acciones}\strut
\end{minipage} & \begin{minipage}[t]{0.74\columnwidth}\raggedright\strut
\begin{enumerate}
\def\labelenumi{\arabic{enumi}.}
\tightlist
\item
  El usuario entra en la vista detalle de una colmena.
\item
  Se listan todas las grabaciones.
\item
  Por cada grabación se da la opción de ver detalle o eliminar.
\item
  Se muestra un botón para iniciar una nueva monitorización.
\end{enumerate}\strut
\end{minipage}\tabularnewline
\begin{minipage}[t]{0.20\columnwidth}\raggedright\strut
\textbf{Postcondición}\strut
\end{minipage} & \begin{minipage}[t]{0.74\columnwidth}\raggedright\strut
El número de grabaciones listado es igual al número de grabaciones de
esa colmena en la base de datos.\strut
\end{minipage}\tabularnewline
\begin{minipage}[t]{0.20\columnwidth}\raggedright\strut
\textbf{Excepciones}\strut
\end{minipage} & \begin{minipage}[t]{0.74\columnwidth}\raggedright\strut
\begin{itemize}
\tightlist
\item
  Error al cargar grabaciones (mensaje).
\item
  No existe ninguna grabación (vista especial).
\end{itemize}\strut
\end{minipage}\tabularnewline
\begin{minipage}[t]{0.20\columnwidth}\raggedright\strut
\textbf{Importancia}\strut
\end{minipage} & \begin{minipage}[t]{0.74\columnwidth}\raggedright\strut
Alta\strut
\end{minipage}\tabularnewline
\bottomrule
\caption{CU-15 Gestión de grabaciones.}
\end{longtable}

\begin{longtable}[H]{@{}ll@{}}
\toprule
\begin{minipage}[b]{0.26\columnwidth}\raggedright\strut
\textbf{CU-16}\strut
\end{minipage} & \begin{minipage}[b]{0.68\columnwidth}\raggedright\strut
\textbf{Añadir grabación}\strut
\end{minipage}\tabularnewline
\midrule
\endhead
\begin{minipage}[t]{0.26\columnwidth}\raggedright\strut
\textbf{Versión}\strut
\end{minipage} & \begin{minipage}[t]{0.68\columnwidth}\raggedright\strut
1.0\strut
\end{minipage}\tabularnewline
\begin{minipage}[t]{0.26\columnwidth}\raggedright\strut
\textbf{Autor}\strut
\end{minipage} & \begin{minipage}[t]{0.68\columnwidth}\raggedright\strut
David Miguel Lozano\strut
\end{minipage}\tabularnewline
\begin{minipage}[t]{0.26\columnwidth}\raggedright\strut
\textbf{Requisitos asociados}\strut
\end{minipage} & \begin{minipage}[t]{0.68\columnwidth}\raggedright\strut
RF-3.1\strut
\end{minipage}\tabularnewline
\begin{minipage}[t]{0.26\columnwidth}\raggedright\strut
\textbf{Descripción}\strut
\end{minipage} & \begin{minipage}[t]{0.68\columnwidth}\raggedright\strut
Permite añadir una nueva grabación a partir de los datos recogidos
durante la monitorización.\strut
\end{minipage}\tabularnewline
\begin{minipage}[t]{0.26\columnwidth}\raggedright\strut
\textbf{Precondición}\strut
\end{minipage} & \begin{minipage}[t]{0.68\columnwidth}\raggedright\strut
La base de datos se encuentra disponible.

El colmenar y la colmena existen.\strut
\end{minipage}\tabularnewline
\begin{minipage}[t]{0.26\columnwidth}\raggedright\strut
\textbf{Acciones}\strut
\end{minipage} & \begin{minipage}[t]{0.68\columnwidth}\raggedright\strut
\begin{enumerate}
\def\labelenumi{\arabic{enumi}.}
\tightlist
\item
  El usuario presiona el botón de finalizar monitorización.
\item
  Si no hay ningún error, se guarda una nueva grabación con los datos
  recogidos durante la monitorización (número de abejas e información
  meteorológica) y se asocia a la colmena.
\item
  Volver a Gestión de grabaciones.
\end{enumerate}\strut
\end{minipage}\tabularnewline
\begin{minipage}[t]{0.26\columnwidth}\raggedright\strut
\textbf{Postcondición}\strut
\end{minipage} & \begin{minipage}[t]{0.68\columnwidth}\raggedright\strut
Existe una grabación más para esa colmena en la base de datos.\strut
\end{minipage}\tabularnewline
\begin{minipage}[t]{0.26\columnwidth}\raggedright\strut
\textbf{Excepciones}\strut
\end{minipage} & \begin{minipage}[t]{0.68\columnwidth}\raggedright\strut
\begin{itemize}
\tightlist
\item
  Error al guardar grabación (mensaje).
\item
  Grabación demasiado corta (mensaje).
\end{itemize}\strut
\end{minipage}\tabularnewline
\begin{minipage}[t]{0.26\columnwidth}\raggedright\strut
\textbf{Importancia}\strut
\end{minipage} & \begin{minipage}[t]{0.68\columnwidth}\raggedright\strut
Alta\strut
\end{minipage}\tabularnewline
\bottomrule
\caption{CU-16 Añadir grabación.}
\end{longtable}

\begin{longtable}[H]{@{}ll@{}}
\toprule
\begin{minipage}[b]{0.28\columnwidth}\raggedright\strut
\textbf{CU-17}\strut
\end{minipage} & \begin{minipage}[b]{0.66\columnwidth}\raggedright\strut
\textbf{Eliminar grabación}\strut
\end{minipage}\tabularnewline
\midrule
\endhead
\begin{minipage}[t]{0.28\columnwidth}\raggedright\strut
\textbf{Versión}\strut
\end{minipage} & \begin{minipage}[t]{0.66\columnwidth}\raggedright\strut
1.0\strut
\end{minipage}\tabularnewline
\begin{minipage}[t]{0.28\columnwidth}\raggedright\strut
\textbf{Autor}\strut
\end{minipage} & \begin{minipage}[t]{0.66\columnwidth}\raggedright\strut
David Miguel Lozano\strut
\end{minipage}\tabularnewline
\begin{minipage}[t]{0.28\columnwidth}\raggedright\strut
\textbf{Requisitos asociados}\strut
\end{minipage} & \begin{minipage}[t]{0.66\columnwidth}\raggedright\strut
RF-3.2\strut
\end{minipage}\tabularnewline
\begin{minipage}[t]{0.28\columnwidth}\raggedright\strut
\textbf{Descripción}\strut
\end{minipage} & \begin{minipage}[t]{0.66\columnwidth}\raggedright\strut
Permite al usuario eliminar una grabación ya existente.\strut
\end{minipage}\tabularnewline
\begin{minipage}[t]{0.28\columnwidth}\raggedright\strut
\textbf{Precondición}\strut
\end{minipage} & \begin{minipage}[t]{0.66\columnwidth}\raggedright\strut
La base de datos se encuentra disponible.

El colmenar y la colmena existen.

La grabación a eliminar existe.\strut
\end{minipage}\tabularnewline
\begin{minipage}[t]{0.28\columnwidth}\raggedright\strut
\textbf{Acciones}\strut
\end{minipage} & \begin{minipage}[t]{0.66\columnwidth}\raggedright\strut
\begin{enumerate}
\def\labelenumi{\arabic{enumi}.}
\tightlist
\item
  El usuario selecciona una grabación para eliminar.
\item
  Se eliminan los datos de esa grabación de la base de datos.
\item
  Se elimina la grabación de la vista.
\item
  Se informa al usuario.
\end{enumerate}\strut
\end{minipage}\tabularnewline
\begin{minipage}[t]{0.28\columnwidth}\raggedright\strut
\textbf{Postcondición}\strut
\end{minipage} & \begin{minipage}[t]{0.66\columnwidth}\raggedright\strut
Existe una grabación menos en esa colmena en la base de datos.\strut
\end{minipage}\tabularnewline
\begin{minipage}[t]{0.28\columnwidth}\raggedright\strut
\textbf{Excepciones}\strut
\end{minipage} & \begin{minipage}[t]{0.66\columnwidth}\raggedright\strut
\begin{itemize}
\tightlist
\item
  Error al eliminar grabación (mensaje).
\end{itemize}\strut
\end{minipage}\tabularnewline
\begin{minipage}[t]{0.28\columnwidth}\raggedright\strut
\textbf{Importancia}\strut
\end{minipage} & \begin{minipage}[t]{0.66\columnwidth}\raggedright\strut
Alta\strut
\end{minipage}\tabularnewline
\bottomrule
\caption{CU-17 Eliminar grabación.}
\end{longtable}

\begin{longtable}[H]{@{}ll@{}}
\toprule
\begin{minipage}[b]{0.26\columnwidth}\raggedright\strut
\textbf{CU-18}\strut
\end{minipage} & \begin{minipage}[b]{0.68\columnwidth}\raggedright\strut
\textbf{Listar grabaciones}\strut
\end{minipage}\tabularnewline
\midrule
\endhead
\begin{minipage}[t]{0.26\columnwidth}\raggedright\strut
\textbf{Versión}\strut
\end{minipage} & \begin{minipage}[t]{0.68\columnwidth}\raggedright\strut
1.0\strut
\end{minipage}\tabularnewline
\begin{minipage}[t]{0.26\columnwidth}\raggedright\strut
\textbf{Autor}\strut
\end{minipage} & \begin{minipage}[t]{0.68\columnwidth}\raggedright\strut
David Miguel Lozano\strut
\end{minipage}\tabularnewline
\begin{minipage}[t]{0.26\columnwidth}\raggedright\strut
\textbf{Requisitos asociados}\strut
\end{minipage} & \begin{minipage}[t]{0.68\columnwidth}\raggedright\strut
RF-3.3\strut
\end{minipage}\tabularnewline
\begin{minipage}[t]{0.26\columnwidth}\raggedright\strut
\textbf{Descripción}\strut
\end{minipage} & \begin{minipage}[t]{0.68\columnwidth}\raggedright\strut
Permite al usuario listar todas las grabaciones de una determinada
colmena. Por cada grabación se muestra la fecha y una previsualización
de la actividad de vuelo.\strut
\end{minipage}\tabularnewline
\begin{minipage}[t]{0.26\columnwidth}\raggedright\strut
\textbf{Precondición}\strut
\end{minipage} & \begin{minipage}[t]{0.68\columnwidth}\raggedright\strut
La base de datos se encuentra disponible.

El colmenar y la colmena existen.\strut
\end{minipage}\tabularnewline
\begin{minipage}[t]{0.26\columnwidth}\raggedright\strut
\textbf{Acciones}\strut
\end{minipage} & \begin{minipage}[t]{0.68\columnwidth}\raggedright\strut
\begin{enumerate}
\def\labelenumi{\arabic{enumi}.}
\tightlist
\item
  El usuario accede a Gestionar Grabaciones de una determinada colmena.
\item
  Se obtienen todas las grabaciones de esa colmena de la base de datos.
\item
  Se muestran la lista de grabaciones. Cada elemento de la lista posee
  la fecha de la grabación y una previsualización de la actividad de
  vuelo.
\end{enumerate}\strut
\end{minipage}\tabularnewline
\begin{minipage}[t]{0.26\columnwidth}\raggedright\strut
\textbf{Postcondición}\strut
\end{minipage} & \begin{minipage}[t]{0.68\columnwidth}\raggedright\strut
-\strut
\end{minipage}\tabularnewline
\begin{minipage}[t]{0.26\columnwidth}\raggedright\strut
\textbf{Excepciones}\strut
\end{minipage} & \begin{minipage}[t]{0.68\columnwidth}\raggedright\strut
\begin{itemize}
\tightlist
\item
  Error al cargar grabaciones (mensaje).
\item
  No existen grabaciones (vista especial).
\end{itemize}\strut
\end{minipage}\tabularnewline
\begin{minipage}[t]{0.26\columnwidth}\raggedright\strut
\textbf{Importancia}\strut
\end{minipage} & \begin{minipage}[t]{0.68\columnwidth}\raggedright\strut
Alta\strut
\end{minipage}\tabularnewline
\bottomrule
\caption{CU-18 Listar grabaciones.}
\end{longtable}

\begin{longtable}[H]{@{}ll@{}}
\toprule
\begin{minipage}[b]{0.26\columnwidth}\raggedright\strut
\textbf{CU-19}\strut
\end{minipage} & \begin{minipage}[b]{0.68\columnwidth}\raggedright\strut
\textbf{Ver grabación}\strut
\end{minipage}\tabularnewline
\midrule
\endhead
\begin{minipage}[t]{0.26\columnwidth}\raggedright\strut
\textbf{Versión}\strut
\end{minipage} & \begin{minipage}[t]{0.68\columnwidth}\raggedright\strut
1.0\strut
\end{minipage}\tabularnewline
\begin{minipage}[t]{0.26\columnwidth}\raggedright\strut
\textbf{Autor}\strut
\end{minipage} & \begin{minipage}[t]{0.68\columnwidth}\raggedright\strut
David Miguel Lozano\strut
\end{minipage}\tabularnewline
\begin{minipage}[t]{0.26\columnwidth}\raggedright\strut
\textbf{Requisitos asociados}\strut
\end{minipage} & \begin{minipage}[t]{0.68\columnwidth}\raggedright\strut
RF-3.4\strut
\end{minipage}\tabularnewline
\begin{minipage}[t]{0.26\columnwidth}\raggedright\strut
\textbf{Descripción}\strut
\end{minipage} & \begin{minipage}[t]{0.68\columnwidth}\raggedright\strut
Permite al usuario visualizar toda la información (actividad de vuelo,
temperatura, precipitaciones y viento) relativa a una determinada
grabación existente.\strut
\end{minipage}\tabularnewline
\begin{minipage}[t]{0.26\columnwidth}\raggedright\strut
\textbf{Precondición}\strut
\end{minipage} & \begin{minipage}[t]{0.68\columnwidth}\raggedright\strut
La base de datos se encuentra disponible.

El colmenar y la colmena existen.

La grabación a visualizar existe.\strut
\end{minipage}\tabularnewline
\begin{minipage}[t]{0.26\columnwidth}\raggedright\strut
\textbf{Acciones}\strut
\end{minipage} & \begin{minipage}[t]{0.68\columnwidth}\raggedright\strut
\begin{enumerate}
\def\labelenumi{\arabic{enumi}.}
\tightlist
\item
  El usuario selecciona una grabación de una determinada colmena para
  visualizar.
\item
  Se obtienen los datos de la grabación de la base de datos (actividad
  de vuelo, temperatura, precipitaciones y viento).
\item
  Se muestra un gráfico con la actividad de vuelo.
\item
  Se muestra un gráfico con la evolución de la temperatura.
\item
  Se muestra un gráfico con la evolución de las precipitaciones.
\item
  Se muestra un gráfico con la evolución del viento.
\end{enumerate}\strut
\end{minipage}\tabularnewline
\begin{minipage}[t]{0.26\columnwidth}\raggedright\strut
\textbf{Postcondición}\strut
\end{minipage} & \begin{minipage}[t]{0.68\columnwidth}\raggedright\strut
-\strut
\end{minipage}\tabularnewline
\begin{minipage}[t]{0.26\columnwidth}\raggedright\strut
\textbf{Excepciones}\strut
\end{minipage} & \begin{minipage}[t]{0.68\columnwidth}\raggedright\strut
\begin{itemize}
\tightlist
\item
  Error al cargar grabación (mensaje).
\end{itemize}\strut
\end{minipage}\tabularnewline
\begin{minipage}[t]{0.26\columnwidth}\raggedright\strut
\textbf{Importancia}\strut
\end{minipage} & \begin{minipage}[t]{0.68\columnwidth}\raggedright\strut
Alta\strut
\end{minipage}\tabularnewline
\bottomrule
\caption{CU-19 Ver grabación.}
\end{longtable}

\begin{longtable}[H]{@{}ll@{}}
\toprule
\begin{minipage}[b]{0.269\columnwidth}\raggedright\strut
\textbf{CU-20}\strut
\end{minipage} & \begin{minipage}[b]{0.75\columnwidth}\raggedright\strut
\textbf{Monitorizar actividad de vuelo}\strut
\end{minipage}\tabularnewline
\midrule
\endhead
\begin{minipage}[t]{0.269\columnwidth}\raggedright\strut
\textbf{Versión}\strut
\end{minipage} & \begin{minipage}[t]{0.75\columnwidth}\raggedright\strut
1.0\strut
\end{minipage}\tabularnewline
\begin{minipage}[t]{0.269\columnwidth}\raggedright\strut
\textbf{Autor}\strut
\end{minipage} & \begin{minipage}[t]{0.75\columnwidth}\raggedright\strut
David Miguel Lozano\strut
\end{minipage}\tabularnewline
\begin{minipage}[t]{0.269\columnwidth}\raggedright\strut
\textbf{Requisitos asociados}\strut
\end{minipage} & \begin{minipage}[t]{0.75\columnwidth}\raggedright\strut
RF-4, RF-4.1, RF-4.2, RF-4.3\strut
\end{minipage}\tabularnewline
\begin{minipage}[t]{0.269\columnwidth}\raggedright\strut
\textbf{Descripción}\strut
\end{minipage} & \begin{minipage}[t]{0.75\columnwidth}\raggedright\strut
Permite al usuario monitorizar la actividad de vuelo de una colmena a
partir de una determinada parametrización.\strut
\end{minipage}\tabularnewline
\begin{minipage}[t]{0.269\columnwidth}\raggedright\strut
\textbf{Precondición}\strut
\end{minipage} & \begin{minipage}[t]{0.75\columnwidth}\raggedright\strut
Se poseen permisos de cámara.

La cámara se encuentra disponible.

El colmenar y la colmena existen.\strut
\end{minipage}\tabularnewline
\begin{minipage}[t]{0.269\columnwidth}\raggedright\strut
\textbf{Acciones}\strut
\end{minipage} & \begin{minipage}[t]{0.75\columnwidth}\raggedright\strut
\begin{enumerate}
\def\labelenumi{\arabic{enumi}.}
\tightlist
\item
  El usuario pulsa el botón de inicializar nueva monitorización.
\item
  Se muestra una previsualización de la salida del algoritmo.
\item
  Si el usuario presiona el botón de configurar:

  \begin{enumerate}
  \def\labelenumii{\alph{enumii}.}
  \tightlist
  \item
    Abrir ajustes.
  \item
    El usuario realiza los ajustes oportunos.
  \item
    Actualizar algoritmo y cámara con los ajustes.
  \item
    Volver a la previsualización.
  \end{enumerate}
\item
  Si el usuario presiona el botón de iniciar monitorización.

  \begin{enumerate}
  \def\labelenumii{\alph{enumii}.}
  \tightlist
  \item
    Se lanza el servicio de monitorización.
  \item
    Se realiza una cuenta atrás de 5 segundos antes de empezar a
    monitorizar.
  \item
    Se inicia la cámara.
  \item
    Se consumen los 10 primeros fotogramas para crear el modelo del
    fondo.
  \item
    Se comienza a monitorizar.
  \end{enumerate}
\item
  Por cada fotograma recibido:

  \begin{enumerate}
  \def\labelenumii{\alph{enumii}.}
  \tightlist
  \item
    Se convierte a escala de grises.
  \item
    Se aplica un desenfoque Gaussiano.
  \item
    Se aplica BackgroundSubtractorMOG2.
  \item
    Se aplican varias fases de erosión y dilatación.
  \item
    Se obtienen los contornos de las regiones en movimiento.
  \item
    Se contabilizan como abejas aquellos contornos que cumplen las
    condiciones.
  \item
    Se almacena el resultado.
  \end{enumerate}
\item
  Si el colmenar posee localización:

  \begin{enumerate}
  \def\labelenumii{\alph{enumii}.}
  \tightlist
  \item
    Cada 15 minutos, obtener información meteorológica.
  \item
    Guardarla en la base de datos asociada al colmenar.
  \end{enumerate}
\item
  Cuando se recibe la orden de finalizar:

  \begin{enumerate}
  \def\labelenumii{\alph{enumii}.}
  \tightlist
  \item
    Cerrar la cámara.
  \item
    Dejar de consultar información meteorológica.
  \item
    Devolver datos recolectados.
  \end{enumerate}
\end{enumerate}\strut
\end{minipage}\tabularnewline
\begin{minipage}[t]{0.269\columnwidth}\raggedright\strut
\textbf{Postcondición}\strut
\end{minipage} & \begin{minipage}[t]{0.75\columnwidth}\raggedright\strut
La grabación tiene más de 5 registros.\strut
\end{minipage}\tabularnewline
\begin{minipage}[t]{0.269\columnwidth}\raggedright\strut
\textbf{Excepciones}\strut
\end{minipage} & \begin{minipage}[t]{0.75\columnwidth}\raggedright\strut
\begin{itemize}
\tightlist
\item
  No se tienen permisos de cámara (solicitar).
\item
  Error de cámara (cancelar).
\item
  No existe conexión a internet (no obtener información meteorológica).
\item
  Error al obtener información meteorológica (ignorar).
\end{itemize}\strut
\end{minipage}\tabularnewline
\begin{minipage}[t]{0.269\columnwidth}\raggedright\strut
\textbf{Importancia}\strut
\end{minipage} & \begin{minipage}[t]{0.75\columnwidth}\raggedright\strut
Alta\strut
\end{minipage}\tabularnewline
\bottomrule
\caption{CU-20 Monitorizar actividad de vuelo.}
\end{longtable}

\begin{longtable}[H]{@{}ll@{}}
\toprule
\begin{minipage}[b]{0.26\columnwidth}\raggedright\strut
\textbf{CU-21}\strut
\end{minipage} & \begin{minipage}[b]{0.68\columnwidth}\raggedright\strut
\textbf{Previsualización del algoritmo}\strut
\end{minipage}\tabularnewline
\midrule
\endhead
\begin{minipage}[t]{0.26\columnwidth}\raggedright\strut
\textbf{Versión}\strut
\end{minipage} & \begin{minipage}[t]{0.68\columnwidth}\raggedright\strut
1.0\strut
\end{minipage}\tabularnewline
\begin{minipage}[t]{0.26\columnwidth}\raggedright\strut
\textbf{Autor}\strut
\end{minipage} & \begin{minipage}[t]{0.68\columnwidth}\raggedright\strut
David Miguel Lozano\strut
\end{minipage}\tabularnewline
\begin{minipage}[t]{0.26\columnwidth}\raggedright\strut
\textbf{Requisitos asociados}\strut
\end{minipage} & \begin{minipage}[t]{0.68\columnwidth}\raggedright\strut
RF-4.1\strut
\end{minipage}\tabularnewline
\begin{minipage}[t]{0.26\columnwidth}\raggedright\strut
\textbf{Descripción}\strut
\end{minipage} & \begin{minipage}[t]{0.68\columnwidth}\raggedright\strut
Permite al usuario previsualizar los resultados que está proporcionando
el algoritmo de conteo en tiempo real (visualización de los fotogramas
de entrada, la máscara de salida y el número de abejas contadas).\strut
\end{minipage}\tabularnewline
\begin{minipage}[t]{0.26\columnwidth}\raggedright\strut
\textbf{Precondición}\strut
\end{minipage} & \begin{minipage}[t]{0.68\columnwidth}\raggedright\strut
Se poseen permisos de cámara.

La cámara se encuentra disponible.

El colmenar y la colmena existen.\strut
\end{minipage}\tabularnewline
\begin{minipage}[t]{0.26\columnwidth}\raggedright\strut
\textbf{Acciones}\strut
\end{minipage} & \begin{minipage}[t]{0.68\columnwidth}\raggedright\strut
\begin{enumerate}
\def\labelenumi{\arabic{enumi}.}
\tightlist
\item
  El usuario pulsa el botón de inicializar nueva monitorización.
\item
  Se muestra en tiempo real los fotogramas (bien los de entrada del
  algoritmo o los de salida).
\item
  Se muestra el número de abejas que contabiliza en cada fotograma
  analizado.
\item
  Si el usuario modifica algún ajuste, se actualiza en la
  previsualización.
\end{enumerate}\strut
\end{minipage}\tabularnewline
\begin{minipage}[t]{0.26\columnwidth}\raggedright\strut
\textbf{Postcondición}\strut
\end{minipage} & \begin{minipage}[t]{0.68\columnwidth}\raggedright\strut
-\strut
\end{minipage}\tabularnewline
\begin{minipage}[t]{0.26\columnwidth}\raggedright\strut
\textbf{Excepciones}\strut
\end{minipage} & \begin{minipage}[t]{0.68\columnwidth}\raggedright\strut
\begin{itemize}
\tightlist
\item
  No se tienen permisos de cámara (solicitar).
\item
  Error de cámara (cancelar).
\end{itemize}\strut
\end{minipage}\tabularnewline
\begin{minipage}[t]{0.26\columnwidth}\raggedright\strut
\textbf{Importancia}\strut
\end{minipage} & \begin{minipage}[t]{0.68\columnwidth}\raggedright\strut
Alta\strut
\end{minipage}\tabularnewline
\bottomrule
\caption{CU-21 Previsualización del algoritmo.}
\end{longtable}

\begin{longtable}[H]{@{}ll@{}}
\toprule
\begin{minipage}[b]{0.26\columnwidth}\raggedright\strut
\textbf{CU-22}\strut
\end{minipage} & \begin{minipage}[b]{0.68\columnwidth}\raggedright\strut
\textbf{Configuración de la monitorización}\strut
\end{minipage}\tabularnewline
\midrule
\endhead
\begin{minipage}[t]{0.26\columnwidth}\raggedright\strut
\textbf{Versión}\strut
\end{minipage} & \begin{minipage}[t]{0.68\columnwidth}\raggedright\strut
1.0\strut
\end{minipage}\tabularnewline
\begin{minipage}[t]{0.26\columnwidth}\raggedright\strut
\textbf{Autor}\strut
\end{minipage} & \begin{minipage}[t]{0.68\columnwidth}\raggedright\strut
David Miguel Lozano\strut
\end{minipage}\tabularnewline
\begin{minipage}[t]{0.26\columnwidth}\raggedright\strut
\textbf{Requisitos asociados}\strut
\end{minipage} & \begin{minipage}[t]{0.68\columnwidth}\raggedright\strut
RF-4.2\strut
\end{minipage}\tabularnewline
\begin{minipage}[t]{0.26\columnwidth}\raggedright\strut
\textbf{Descripción}\strut
\end{minipage} & \begin{minipage}[t]{0.68\columnwidth}\raggedright\strut
Permite al usuario configurar todos los parámetros relativos a la
monitorización (parámetros del algoritmo y parámetros de la
cámara).\strut
\end{minipage}\tabularnewline
\begin{minipage}[t]{0.26\columnwidth}\raggedright\strut
\textbf{Precondición}\strut
\end{minipage} & \begin{minipage}[t]{0.68\columnwidth}\raggedright\strut
Se poseen permisos de cámara.

La cámara se encuentra disponible.

El colmenar y la colmena existen.\strut
\end{minipage}\tabularnewline
\begin{minipage}[t]{0.26\columnwidth}\raggedright\strut
\textbf{Acciones}\strut
\end{minipage} & \begin{minipage}[t]{0.68\columnwidth}\raggedright\strut
\begin{enumerate}
\def\labelenumi{\arabic{enumi}.}
\tightlist
\item
  El usuario se encuentra en la pantalla de previsualización y pulsa el
  botón de ajustes.
\item
  Se abre una ventana con los diferentes parámetros ajustables (mostrar
  salida o entrada del algoritmo, modificar tamaño de las regiones,
  ajustar áreas de una abeja, zoom y frecuencia de muestreo).
\item
  Cuando el usuario realiza alguna modificación, actualizar
  instantáneamente ese parámetro en la cámara o en el algoritmo.
\end{enumerate}\strut
\end{minipage}\tabularnewline
\begin{minipage}[t]{0.26\columnwidth}\raggedright\strut
\textbf{Postcondición}\strut
\end{minipage} & \begin{minipage}[t]{0.68\columnwidth}\raggedright\strut
-\strut
\end{minipage}\tabularnewline
\begin{minipage}[t]{0.26\columnwidth}\raggedright\strut
\textbf{Excepciones}\strut
\end{minipage} & \begin{minipage}[t]{0.68\columnwidth}\raggedright\strut
\begin{itemize}
\tightlist
\item
  No se tienen permisos de cámara (solicitar).
\item
  Error de cámara (cancelar).
\end{itemize}\strut
\end{minipage}\tabularnewline
\begin{minipage}[t]{0.26\columnwidth}\raggedright\strut
\textbf{Importancia}\strut
\end{minipage} & \begin{minipage}[t]{0.68\columnwidth}\raggedright\strut
Alta\strut
\end{minipage}\tabularnewline
\bottomrule
\caption{CU-22 Configuración de la monitorización.}
\end{longtable}

\begin{longtable}[H]{@{}ll@{}}
\toprule
\begin{minipage}[b]{0.26\columnwidth}\raggedright\strut
\textbf{CU-23}\strut
\end{minipage} & \begin{minipage}[b]{0.68\columnwidth}\raggedright\strut
\textbf{Configuración de la aplicación}\strut
\end{minipage}\tabularnewline
\midrule
\endhead
\begin{minipage}[t]{0.26\columnwidth}\raggedright\strut
\textbf{Versión}\strut
\end{minipage} & \begin{minipage}[t]{0.68\columnwidth}\raggedright\strut
1.0\strut
\end{minipage}\tabularnewline
\begin{minipage}[t]{0.26\columnwidth}\raggedright\strut
\textbf{Autor}\strut
\end{minipage} & \begin{minipage}[t]{0.68\columnwidth}\raggedright\strut
David Miguel Lozano\strut
\end{minipage}\tabularnewline
\begin{minipage}[t]{0.26\columnwidth}\raggedright\strut
\textbf{Requisitos asociados}\strut
\end{minipage} & \begin{minipage}[t]{0.68\columnwidth}\raggedright\strut
RF-5\strut
\end{minipage}\tabularnewline
\begin{minipage}[t]{0.26\columnwidth}\raggedright\strut
\textbf{Descripción}\strut
\end{minipage} & \begin{minipage}[t]{0.68\columnwidth}\raggedright\strut
Permite al usuario configurar todos los parámetros disponibles en la
aplicación, como el idioma o las unidades meteorológicas o realizar
determinadas tareas de mantenimiento.\strut
\end{minipage}\tabularnewline
\begin{minipage}[t]{0.26\columnwidth}\raggedright\strut
\textbf{Precondición}\strut
\end{minipage} & \begin{minipage}[t]{0.68\columnwidth}\raggedright\strut
La base de datos se encuentra disponible.\strut
\end{minipage}\tabularnewline
\begin{minipage}[t]{0.26\columnwidth}\raggedright\strut
\textbf{Acciones}\strut
\end{minipage} & \begin{minipage}[t]{0.68\columnwidth}\raggedright\strut
\begin{enumerate}
\def\labelenumi{\arabic{enumi}.}
\tightlist
\item
  El usuario presiona el botón de ajustes de aplicación.
\item
  Se abre una ventana con los diferentes parámetros ajustables.
\item
  Si el usuario modifica cualquier parámetro, se hace efectiva la nueva
  configuración al instante.
\end{enumerate}\strut
\end{minipage}\tabularnewline
\begin{minipage}[t]{0.26\columnwidth}\raggedright\strut
\textbf{Postcondición}\strut
\end{minipage} & \begin{minipage}[t]{0.68\columnwidth}\raggedright\strut
-\strut
\end{minipage}\tabularnewline
\begin{minipage}[t]{0.26\columnwidth}\raggedright\strut
\textbf{Excepciones}\strut
\end{minipage} & \begin{minipage}[t]{0.68\columnwidth}\raggedright\strut
\begin{itemize}
\tightlist
\item
  Error al guardar configuración (mensaje).
\end{itemize}\strut
\end{minipage}\tabularnewline
\begin{minipage}[t]{0.26\columnwidth}\raggedright\strut
\textbf{Importancia}\strut
\end{minipage} & \begin{minipage}[t]{0.68\columnwidth}\raggedright\strut
Alta\strut
\end{minipage}\tabularnewline
\bottomrule
\caption{CU-23 Configuración de la aplicación.}
\end{longtable}

\begin{longtable}[H]{@{}ll@{}}
\toprule
\begin{minipage}[b]{0.20\columnwidth}\raggedright\strut
\textbf{CU-24}\strut
\end{minipage} & \begin{minipage}[b]{0.74\columnwidth}\raggedright\strut
\textbf{Ayuda de la aplicación}\strut
\end{minipage}\tabularnewline
\midrule
\endhead
\begin{minipage}[t]{0.20\columnwidth}\raggedright\strut
\textbf{Versión}\strut
\end{minipage} & \begin{minipage}[t]{0.74\columnwidth}\raggedright\strut
1.0\strut
\end{minipage}\tabularnewline
\begin{minipage}[t]{0.20\columnwidth}\raggedright\strut
\textbf{Autor}\strut
\end{minipage} & \begin{minipage}[t]{0.74\columnwidth}\raggedright\strut
David Miguel Lozano\strut
\end{minipage}\tabularnewline
\begin{minipage}[t]{0.20\columnwidth}\raggedright\strut
\textbf{Requisitos asociados}\strut
\end{minipage} & \begin{minipage}[t]{0.74\columnwidth}\raggedright\strut
RF-6\strut
\end{minipage}\tabularnewline
\begin{minipage}[t]{0.20\columnwidth}\raggedright\strut
\textbf{Descripción}\strut
\end{minipage} & \begin{minipage}[t]{0.74\columnwidth}\raggedright\strut
Permite al usuario obtener ayuda sobre cada una de las funcionalidades
de la aplicación.\strut
\end{minipage}\tabularnewline
\begin{minipage}[t]{0.20\columnwidth}\raggedright\strut
\textbf{Precondición}\strut
\end{minipage} & \begin{minipage}[t]{0.74\columnwidth}\raggedright\strut
Se dispone de permisos de internet.

Se dispone de conexión a internet.\strut
\end{minipage}\tabularnewline
\begin{minipage}[t]{0.20\columnwidth}\raggedright\strut
\textbf{Acciones}\strut
\end{minipage} & \begin{minipage}[t]{0.74\columnwidth}\raggedright\strut
\begin{enumerate}
\def\labelenumi{\arabic{enumi}.}
\tightlist
\item
  El usuario presiona el botón de ayuda de aplicación.
\item
  Se abre una ventana que carga una página web con la ayuda de la
  aplicación categorizada por acciones.
\end{enumerate}\strut
\end{minipage}\tabularnewline
\begin{minipage}[t]{0.20\columnwidth}\raggedright\strut
\textbf{Postcondición}\strut
\end{minipage} & \begin{minipage}[t]{0.74\columnwidth}\raggedright\strut
-\strut
\end{minipage}\tabularnewline
\begin{minipage}[t]{0.20\columnwidth}\raggedright\strut
\textbf{Excepciones}\strut
\end{minipage} & \begin{minipage}[t]{0.74\columnwidth}\raggedright\strut
\begin{itemize}
\tightlist
\item
  No se disponen de permisos de internet (solicitar),
\item
  No hay conexión a internet (mensaje).
\end{itemize}\strut
\end{minipage}\tabularnewline
\begin{minipage}[t]{0.20\columnwidth}\raggedright\strut
\textbf{Importancia}\strut
\end{minipage} & \begin{minipage}[t]{0.74\columnwidth}\raggedright\strut
Alta\strut
\end{minipage}\tabularnewline
\bottomrule
\caption{CU-24 Ayuda de la aplicación.}
\end{longtable}

\begin{longtable}[H]{@{}ll@{}}
\toprule
\begin{minipage}[b]{0.26\columnwidth}\raggedright\strut
\textbf{CU-25}\strut
\end{minipage} & \begin{minipage}[b]{0.68\columnwidth}\raggedright\strut
\textbf{Información de la aplicación}\strut
\end{minipage}\tabularnewline
\midrule
\endhead
\begin{minipage}[t]{0.26\columnwidth}\raggedright\strut
\textbf{Versión}\strut
\end{minipage} & \begin{minipage}[t]{0.68\columnwidth}\raggedright\strut
1.0\strut
\end{minipage}\tabularnewline
\begin{minipage}[t]{0.26\columnwidth}\raggedright\strut
\textbf{Autor}\strut
\end{minipage} & \begin{minipage}[t]{0.68\columnwidth}\raggedright\strut
David Miguel Lozano\strut
\end{minipage}\tabularnewline
\begin{minipage}[t]{0.26\columnwidth}\raggedright\strut
\textbf{Requisitos asociados}\strut
\end{minipage} & \begin{minipage}[t]{0.68\columnwidth}\raggedright\strut
RF-7\strut
\end{minipage}\tabularnewline
\begin{minipage}[t]{0.26\columnwidth}\raggedright\strut
\textbf{Descripción}\strut
\end{minipage} & \begin{minipage}[t]{0.68\columnwidth}\raggedright\strut
Permite al usuario obtener información sobre la aplicación, compartirla
o enviar sugerencias.\strut
\end{minipage}\tabularnewline
\begin{minipage}[t]{0.26\columnwidth}\raggedright\strut
\textbf{Precondición}\strut
\end{minipage} & \begin{minipage}[t]{0.68\columnwidth}\raggedright\strut
-\strut
\end{minipage}\tabularnewline
\begin{minipage}[t]{0.26\columnwidth}\raggedright\strut
\textbf{Acciones}\strut
\end{minipage} & \begin{minipage}[t]{0.68\columnwidth}\raggedright\strut
\begin{enumerate}
\def\labelenumi{\arabic{enumi}.}
\tightlist
\item
  Si el usuario presiona el botón de compartir aplicación, se le
  muestran los diferentes medios soportados por el dispositivo para
  compartirla.
\item
  Si el usuario presiona sobre el botón de enviar comentarios, se abre
  la aplicación de email con la información del destinatario rellenada
  para que el usuario pueda enviar sus sugerencias.
\item
  Si el usuario presiona sobre el botón acerca de GoBees, se abre una
  ventana con información sobre la versión, autor, licencia, página web,
  historial de cambios y librerías utilizadas junto con sus licencias.
\end{enumerate}\strut
\end{minipage}\tabularnewline
\begin{minipage}[t]{0.26\columnwidth}\raggedright\strut
\textbf{Postcondición}\strut
\end{minipage} & \begin{minipage}[t]{0.68\columnwidth}\raggedright\strut
-\strut
\end{minipage}\tabularnewline
\begin{minipage}[t]{0.26\columnwidth}\raggedright\strut
\textbf{Excepciones}\strut
\end{minipage} & \begin{minipage}[t]{0.68\columnwidth}\raggedright\strut
-\strut
\end{minipage}\tabularnewline
\begin{minipage}[t]{0.26\columnwidth}\raggedright\strut
\textbf{Importancia}\strut
\end{minipage} & \begin{minipage}[t]{0.68\columnwidth}\raggedright\strut
Media\strut
\end{minipage}\tabularnewline
\bottomrule
\caption{CU-25 Información de la aplicación.}
\end{longtable}

\apendice{Especificación de diseño}

\section{Introducción}\label{introduccion}

En este anexo se define cómo se han resuelto los objetivos y
especificaciones expuestos con anterioridad. Define los datos que va a
manejar la aplicación, su arquitectura, el diseño de sus interfaces, sus
detalles procedimentales, etc.

\section{Diseño de datos}\label{diseno-de-datos}

La aplicación cuenta con las siguientes entidades:

\begin{itemize}
\tightlist
\item
  \textbf{Colmenar (Apiary)}: tiene un nombre, una imagen, una
  localización y unas notas. A su vez, guarda un registro del tiempo
  meteorológico actual y varios registros del tiempo que hacía cuando se
  realizaron las grabaciones de sus colmenas.
\item
  \textbf{Colmena (Hive)}: tiene un nombre, una imagen y unas notas. A
  su vez, posee varias grabaciones de distintas monitorizaciones de la
  colmena.
\item
  \textbf{Registro (Record)}: se corresponde a la salida del algoritmo
  de conteo al analizar un fotograma. Tiene un \emph{timestamp} y el
  número de abejas que había en el fotograma.
\item
  \textbf{Registro meteorológico (MeteoRecord)}: guarda información
  sobre el estado meteorológico en una localización y un momento dado.
  Tiene un \emph{timestamp}, la localidad, el código correspondiente a
  la condición meteorológica, el icono correspondiente, temperatura,
  presión, humedad, velocidad y dirección del viento, porcentaje de
  nubes, precipitaciones, y nieve.
\end{itemize}
\newpage
\subsection{Diagrama E/R}\label{diagrama-er}

\imagenAncho{er-diagram}{Diagrama E/R.}{1}

\subsection{Diagrama Relacional}\label{diagrama-relacional}

\imagenAncho{relational-diagram}{Diagrama relacional.}{0.8}

\section{Diseño arquitectónico}\label{diseno-arquitectonico}

El hecho de que el proyecto se haya realizado para la plataforma Android
ha condicionado muchas de las decisiones de diseño. Aun así, se han
aplicado una serie de patrones para intentar desacoplar el código lo
máximo posible y así mejorar su testabilidad y mantenibilidad.

\subsection{Model-View-Presenter (MVP)}\label{model-view-presenter-mvp}

Uno de los patrones arquitectónicos que más relevancia está ganando para
el desarrollo de aplicaciones es MVP (\emph{Model-View-Presenter)}. Se
trata de un patrón derivado del MVC (\emph{Model-View-Controler}) cuyo
objetivo es separar la vista del modelo de datos subyacente. MVP
introduce la figura del \emph{presenter} que actúa de mediador entre
estas dos capas. Su segundo objetivo es maximizar la cantidad de código
que se puede testear de forma automática.

MVP divide la aplicación en las siguientes capas:\citep{pattern:mvp}

\begin{itemize}
\tightlist
\item
  \emph{Model}: se corresponde únicamente con el acceso a datos. Se
  encarga de almacenar y proporcionar los diferentes datos que maneja la
  aplicación. En nuestra aplicación se corresponde con el Repositorio.
\item
  \emph{View}: se encarga de la visualización de los datos (del modelo).
  Propaga todas las acciones de usuario al \emph{presenter}. En nuestra
  aplicación se corresponde con los \emph{Fragments}.
\item
  \emph{Presenter}: enlaza las dos capas anteriores. Sincroniza los
  datos mostrados en la vista con los almacenados en el modelo y actúa
  ante los eventos de usuario propagados por la vista. En nuestra
  aplicación se corresponde con los \emph{Presenters}.
\end{itemize}

\imagen{mvp}{Patrón MVP.}

Existen varias variantes sobre cómo implementar MVP en Android. En
nuestro caso, se ha seguido la expuesta Google en Android Architecture
Blueprints \citep{pattern:android_architecture}. En ella se realizan
las siguientes consideraciones:

\begin{itemize}
\tightlist
\item
  Se utilizan las \emph{Activity} como controladores globales que se
  encargan de crear y conectar las vistas con los \emph{presenters}.
\item
  Se utilizan los \emph{Fragment} como vistas ya que proporcionan
  numerosas ventajas cuando se trabaja con múltiples vistas.
\end{itemize}

\subsection{Patrón repositorio}\label{patron-repositorio}

Para la capa del modelo, se ha utilizado el patrón repositorio que
proporciona una abstracción de la implementación del acceso a datos con
el objetivo de que este sea transparente a la lógica de negocio
\citep{pattern:repository}.

En nuestra aplicación existen dos fuentes de datos: por una parte, está
la base de datos local implementada con Realm, y por otra, tenemos la
API remota que nos da acceso a la información meteorológica. Ambas
fuentes son transparentes para los \emph{presenters}.

El repositorio media entre la capa de acceso a datos y la lógica de
negocio de tal forma que no existe ninguna dependencia entre ellas.
Consiguiendo desacoplar, mantener y testear más fácilmente el código y
permitiendo la reutilización del acceso a datos desde cualquier cliente.

\imagen{repository_pattern}{Patrón repositorio.}

\subsection{Inyección de
dependencias}\label{inyeccion-de-dependencias}

A la hora de testear, es muy frecuente necesitar sustituir la
implementación de una clase por otra ``falsa'' que se comporte de una
manera predeterminada para conseguir probar la funcionalidad de manera
aislada. En nuestro caso, para facilitar la labor de testeo nos vimos
obligados a sustituir la base de datos Realm por una base de datos en
memoria. Esta sustitución se realizó mediante la inyección de
dependencias.

La inyección de dependencias es un patrón mediante el cual se
proporcionan todas las dependencias que una clase necesita para su
funcionamiento, en lugar de ser la propia clase quien las cree. Al
separar las dependencias de la propia clase, se posibilita la opción de
sustituir estas por dobles con un comportamiento definido
\citep{wiki:injection}.

Para la implementación de la inyección de dependencias se han utilizado
los \emph{build flavors} que proporciona Gradle. Se crearon dos
\emph{flavors}:

\begin{itemize}
\tightlist
\item
  \texttt{mock}: inyectaba una base de datos en memoria utilizada para el testeo
  de la aplicación.
\item
  \texttt{prod}: inyectaba la base de datos Realm utilizada para producción.
\end{itemize}

\subsection{Arquitectura general}\label{arquitectura-general}

El resultado de la arquitectura tras aplicar los patrones explicados es
el siguiente:

\imagen{architecture}{Arquitectura de la aplicación.}

Para agilizar la navegación por la aplicación se implementó una capa de
caché en el repositorio.

\subsection{Diseño de paquetes}\label{diseno-de-paquetes}

Para la organización de los diferentes archivos que componen la
aplicación no se utilizó la estrategia convencional de paquete por capa
(\emph{package by layer approach}), sino una estrategia de paquete por
característica (\emph{package per feature approach}).

Siguiendo esta estrategia se agruparon todos los archivos relacionados
cada una de las distintas funcionalidades de la aplicación en un mismo
paquete. De esta manera se mejora notablemente la legibilidad y la
modularización de la aplicación, ya que se puede modificar cada
funcionalidad de forma independiente.

Existen dos paquetes excepcionales que no siguen esta convención:

\begin{itemize}
\tightlist
\item
  Paquete \texttt{data}: agrupa toda la capa de modelo.
\item
  Paquete \texttt{utils}: reúne un conjunto de clases de utilidad
  generales que son utilizadas por varias características.
\end{itemize}

El diagrama de paquetes es el siguiente:

\imagenAncho{packages-diagram}{Diagrama de paquetes simplificado.}{1}

El paquete \texttt{feature X} se correspondería con cada paquete de cada
funcionalidad. Se ha representado de esta manera para simplificar el
diagrama.

A continuación, se muestran por separado los paquetes de todas las
funcionalidades:

\imagenAncho{packages-features-diagram}{Paquetes de las diferentes características.}{1}

\begin{itemize}
\tightlist
\item
  \texttt{about}: contiene la funcionalidad de ``Acerca de'' de la
  aplicación. Donde se muestra el autor, licencia, versión de la
  \emph{app}, sitio web, historial de cambio y todas las dependencias
  junto con sus licencias.
\item
  \texttt{addeditapiary}: permite añadir o editar colmenares.
\item
  \texttt{addedithive}: permite añadir o editar colmenas.
\item
  \texttt{apiaries}: permite listar los colmenares y gestionarlos.
\item
  \texttt{apiary}: permite listar las colmenas de un colmenar,
  gestionarlas y mostrar la información relativa al colmenar.
\item
  \texttt{help}: muestra la ayuda de la aplicación.
\item
  \texttt{hive}: permite listar las grabaciones de una colmena,
  gestionarlas y mostrar la información relativa a la colmena.
\item
  \texttt{monitoring}: agrupa toda la funcionalidad de monitorización de
  la actividad de vuelo de una colmena, desde la configuración hasta la
  ejecución del algoritmo.
\item
  \texttt{recording}: permite visualizar los detalles de una determinada
  grabación.
\item
  \texttt{settings}: permite configurar los distintos parámetros de la
  aplicación.
\item
  \texttt{splash}: muestra una pantalla de inicio mientras la aplicación
  carga en memoria los recursos necesarios.
\end{itemize}

\subsection{Diseño de clases}\label{diseno-de-clases}

Aplicando MVP, cada característica clave de la aplicación posee los
siguientes componentes:

\begin{itemize}
\tightlist
\item
  \texttt{FeatureActivity}: funciona como un controlador global que crea la vista
  y el \emph{presenter} y los enlaza.
\item
  \texttt{FeatureContract}: se trata de una interfaz que establece los siguientes
  contratos:

  \begin{itemize}
  \tightlist
  \item
    \texttt{FeatureContract.View}: define la capa \emph{view} para esta
    característica (las únicas funciones que expone a otras capas).
  \item
    \texttt{FeatureContract.Presenter}: define la interacción entre las capas
    \emph{view} y \emph{presenter}. Describe las acciones que pueden ser
    iniciadas desde la vista.
  \end{itemize}
\item
  \texttt{FeatureFragment}: implementación concreta de la capa \emph{view}.
\item
  \texttt{FeaturePresenter}: implementación concreta de la capa \emph{presenter}.
  Escucha las acciones de usuario y actualiza la vista cuando cambia el
  modelo.
\end{itemize}

\imagenAncho{feature-package}{Paquete tipo de una característica.}{0.5}

El diagrama de clases general que muestra cómo se relacionan todos los
componentes de una determinada característica es el siguiente:

\imagenAncho{general-class-diagram}{Diagrama de clases general.}{1}

El único paquete que se diferencia de la estructura expuesta es el
paquete \texttt{monitoring.} Este integra a su vez toda la lógica de
acceso a la cámara y todas las clases relacionadas con el algoritmo de
conteo.

\imagen{monitoring-package}{Paquete \emph{monitoring}.}

El diagrama de clases del paquete \texttt{camera} es el siguiente:

\imagen{camera-class-diagram}{Diagrama de clases del paquete \emph{camera}.}
\newpage
El diagrama de las clases que implementan el algoritmo de conteo es el
siguiente:

\imagenAncho{algorithm-class-diagram}{Diagrama de clases del paquete \emph{algorithm}.}{1}
\newpage
En la parte del acceso a datos, se poseen dos paquetes como se ha visto
en el apartado anterior.

\imagenAncho{data-package}{Paquete \emph{data}.}{1}

El paquete \emph{model} contiene todas las clases de modelo que se
mapean con la base de datos.

\imagenAncho{model-package}{Paquete \emph{model}.}{0.8}
\newpage
\textbf{Nota:} la clase \texttt{Recording} se utiliza para agrupar a un conjunto de \texttt{Records},
pero no se almacena en la base de datos directamente (solo los Records).

Por otro lado, el paquete \texttt{source} contiene todas las clases
correspondientes a los accesos de las diferentes fuentes de datos. Su
diagrama de clases es el siguiente:

\imagenAncho{source-class-diagram}{Diagrama de clases del paquete \texttt{source}.}{1}

Para conocer a mayor detalle las funciones de cada clase se puede
consultar la documentación JavaDoc de la aplicación.

\section{Diseño procedimental}\label{diseno-procedimental}

En este apartado se recogen los detalles más relevantes respecto a la
ejecución del algoritmo de monitorización de la actividad de vuelo de
una colmena.

En el siguiente diagrama de secuencia se ha representado como es la
interacción entre los diferentes objetos que se encargan de la
inicialización de la monitorización, la obtención de las imágenes y su
posterior procesado por el algoritmo de conteo.

\begin{landscape}
\imagenAncho{algo-sequence-diagram}{Diagrama de secuencia del algoritmo.}{1.5}
\end{landscape}

\section{Diseño de interfaces}\label{diseno-de-interfaces}

En el diseño de la interfaz se ha seguido la guía de estilos de
\emph{Material Design} \citep{design:material} introducida en el Google
I/O 2014 y que se adoptó en Android a partir de la versión 5.0
(\emph{Lollipop}).

En las primeras etapas de proyecto se realizaron una serie de prototipos
en los que se plasmaron las principales funcionalidades de la
aplicación.

\imagenAncho{prototipos}{Prototipos iniciales.}{1}

Tras una serie de iteraciones, estos se fueron mejorando hasta obtener
las interfaces con las que cuenta hoy en día la \emph{app}.

\imagenAncho{features}{Diseños finales de las interfaces.}{1}

El siguiente diagrama muestra la navegabilidad por la aplicación. Esta
ha sido distribuida de acuerdo al tipo de contenido y a las tareas a
realizar sobre este.

\imagenAncho{navegation-diagram}{Diagrama de navegabilidad.}{1}

Se ha escogido la paleta de colores entre los recomendados por
\emph{Material Design}. Utilizando como principal un color en la gama de
los 500, lo que denominan un color \emph{material,} y definiendo otro
color que contraste con este para acentuar.

\imagen{palette}{Paleta de colores.}

\apendice{Documentación técnica de programación}

\section{Introducción}

\section{Estructura de directorios}

\section{Manual del programador}

\section{Compilación, instalación y ejecución del proyecto}

\section{Pruebas del sistema}

\apendice{Documentación de usuario}

\section{Introducción}\label{introduccion-usuario}

En este manual se detallan los requerimientos de la aplicación, cómo
instalarla en un dispositivo Android e indicaciones sobre cómo
utilizarla correctamente. Todos los procedimientos aquí descritos se
encuentran también disponibles en formato video.

\section{Requisitos de usuarios}\label{requisitos-de-usuarios}

Los requisitos mínimos para poder hacer uso de la aplicación son:

\begin{itemize}
\tightlist
\item
  Contar con un dispositivo que posea Android 4.4 (\emph{KitKat} -- API
  19) o superior.
\item
  Para utilizar la característica de monitorización de la actividad, es
  necesario tener instalada la aplicación
  \href{https://play.google.com/store/apps/details?id=org.opencv.engine}{OpenCV
  Manager}.
\item
  También se necesita contar con permiso para acceder a la cámara del
  dispositivo.
\item
  Si se desea localizar los colmenares mediante GPS, es necesario contar
  con un dispositivo que lo soporte y conceder el permiso de
  localización a la aplicación.
\item
  Para acceder a la información meteorológica se requiere conexión a
  internet.
\end{itemize}

\section{Instalación}\label{instalacion}

La instalación se puede realizar de dos maneras: a través de Google Play
o instalando directamente el ejecutable de la aplicación en nuestro
dispositivo.

\subsection{Desde Google Play}\label{desde-google-play}

Google Play es una plataforma de distribución digital de aplicaciones
móviles para los dispositivos Android. GoBees se distribuye por esta
plataforma desde su versión 1.0.

\imagen{gobees-google-play}{GoBees en Google Play.}

Video-tutorial:
\url{http://gobees.io/help/videos/instalacion-google-play}

Para instalar la aplicación debemos realizar los siguientes pasos:

\begin{enumerate}
\def\labelenumi{\arabic{enumi}.}
\tightlist
\item
  Acceder a la aplicación Google Play.
\item
  Buscar el término ``GoBees''.
\item
  Entrar en la sección correspondiente a la aplicación.
\item
  Pulsar el botón instalar.
\item
  Cuando la instalación haya finalizado, pulsar sobre el botón abrir.
\item
  La instalación habrá finalizado y la aplicación estará lista para su
  uso.
\end{enumerate}

\imagenAncho{gobees-google-play-install}{Instalación desde Google Play}{0.5}

\subsection{Desde fichero ejecutable}\label{desde-fichero-ejecutable}

La otra opción, es realizar la instalación directamente desde el fichero
ejecutable de la aplicación. Estos ficheros poseen la extensión
\texttt{.apk}. Podemos conseguir la última versión del \texttt{.apk} de
GoBees desde \citep{github:gobees_apk}.

Video-tutorial: \url{http://gobees.io/help/videos/instalacion-apk}

Una vez descargado, tenemos que seguir los siguientes pasos:

\begin{enumerate}
\def\labelenumi{\arabic{enumi}.}
\tightlist
\item
  En primer lugar, hay que permitir la instalación de ``aplicaciones con
  orígenes desconocidos''. Para ello:

  \begin{enumerate}
  \def\labelenumii{\alph{enumii}.}
  \tightlist
  \item
    Ir a ajustes del dispositivo.
  \item
    Seguridad (o Privacidad).
  \item
    Activar ``Orígenes desconocidos''.
  \end{enumerate}
\item
  Ejecutar el fichero descargado.
\item
  Pulsar el botón instalar.
\item
  Cuando la instalación haya finalizado, pulsar sobre el botón abrir.
\item
  La instalación habrá finalizado y la aplicación estará lista para su
  uso.
\end{enumerate}

\section{Manual de usuario}\label{manual-de-usuario-1}

En esta sección se describe el uso de las diferentes funcionalidades de
la aplicación.

\subsection{Generar datos de muestra}\label{generar-datos-de-muestra}

Una de las mejores maneras de aprender a utilizar una aplicación es
indagando en ella. GoBees permite generar un colmenar de prueba, de tal
manera, que podemos explorar las diferentes secciones con datos reales.

Video-tutorial:
\url{http://gobees.io/help/videos/generar-colmenar-prueba}

Para generar los datos de prueba:

\begin{enumerate}
\def\labelenumi{\arabic{enumi}.}
\tightlist
\item
  Pulsar el botón menú.
\item
  Entrar en la sección ``Ajustes''.
\item
  Seleccionar la opción ``Generar datos de muestra''.
\item
  Se generará un colmenar con tres colmenas y tres grabaciones por
  colmena.
\end{enumerate}

\imagenAncho{sample-apiary}{Colmenar de muestra.}{0.5}

\subsection{Añadir un colmenar}\label{auxf1adir-un-colmenar}

Un colmenar hace referencia al lugar o recinto donde se poseen un
conjunto de colmenas. Un colmenar posee un nombre, una localización y
unas notas.

Video-tutorial: \url{http://gobees.io/help/videos/anadir-colmenar}

Para añadir un nuevo colmenar:

\begin{enumerate}
\def\labelenumi{\arabic{enumi}.}
\tightlist
\item
  Desde la pantalla principal.
\item
  Pulsar el botón ``+''.
\item
  Definir el nombre del colmenar (obligatorio).
\item
  Definir la localización del colmenar (opcional).

  \begin{enumerate}
  \def\labelenumii{\alph{enumii}.}
  \tightlist
  \item
    Se pueden introducir manualmente las coordenadas, indicando la
    latitud y la longitud en el sistema de coordenadas geográficas.
  \item
    Alternativamente, se puede obtener la localización actual
    automáticamente pulsando el botón situado en la parte derecha (se
    necesitan permisos de localización para utilizar esta
    característica).
  \end{enumerate}
\item
  Definir unas notas sobre el colmenar (opcional). En las notas se puede
  apuntar cualquier cosa relacionada con el colmenar en general.
\item
  Pulsar el botón {$\checkmark$} para guardar el nuevo colmenar.
\end{enumerate}

\imagenAncho{add-apiary}{Añadir colmenar.}{0.5}

\subsection{Editar un colmenar}\label{editar-un-colmenar}

Los detalles de un colmenar se pueden editar en cualquier momento.

Video-tutorial: \url{http://gobees.io/help/videos/editar-colmenar}

Para editar un colmenar existente:

\begin{enumerate}
\def\labelenumi{\arabic{enumi}.}
\tightlist
\item
  Desde la pantalla principal.
\item
  Pulsar el botón de menú asociado al colmenar a editar (tres puntos
  verticales situados en la esquina superior derecha).
\item
  Seleccionar la opción de editar.
\item
  Se abrirá la pantalla de edición, donde se podrán modificar los datos
  que se deseen.
\item
  Pulsar el botón {$\checkmark$} para actualizar los datos editados.
\end{enumerate}

\subsection{Eliminar un colmenar}\label{eliminar-un-colmenar}

Al eliminar un colmenar, se eliminan también todos los datos asociados a
este (información del colmenar, colmenas, grabaciones e información
meteorológica).

Video-tutorial: \url{http://gobees.io/help/videos/eliminar-colmenar}

Para eliminar un colmenar existente:

\begin{enumerate}
\def\labelenumi{\arabic{enumi}.}
\tightlist
\item
  Desde la pantalla principal.
\item
  Pulsar el botón de menú asociado al colmenar a eliminar (tres puntos
  verticales situados en la esquina superior derecha).
\item
  Seleccionar la opción de eliminar.
\item
  El colmenar se eliminará junto con toda su información.
\end{enumerate}

\subsection{Consultar la información meteorológica de un
colmenar}\label{consultar-la-informaciuxf3n-meteoroluxf3gica-de-un-colmenar}

Para poder consultar la información meteorológica de un colmenar se
necesita que este posea una localización y que el dispositivo esté
conectado a internet. Si se cumplen estos dos requisitos, la información
meteorológica del colmenar se actualizará automáticamente de forma
periódica.

Video-tutorial:
\url{http://gobees.io/help/videos/consultar-info-meteo-colmenar}

Para consultar la información meteorológica:

\begin{enumerate}
\def\labelenumi{\arabic{enumi}.}
\tightlist
\item
  Asegurarse de que el colmenar tiene definida una localización y que se
  posee conexión a internet.
\item
  En la lista de colmenares, se puede visualizar un resumen con la
  temperatura y situación meteorológica en cada colmenar.
\item
  Si se desea consultar la información en detalle, entrar en el colmenar
  a consultar.
\item
  Desplazarse a la pestaña ``info''.
\item
  En la parte inferior podremos visualizar todos los detalles de la
  situación meteorológica actual en ese colmenar.
\end{enumerate}

Se pueden cambiar las unidades meteorológicas, para ello:

\begin{enumerate}
\def\labelenumi{\arabic{enumi}.}
\tightlist
\item
  En la pantalla principal.
\item
  Pulsar el botón menú.
\item
  Entrar en la sección ``Ajustes''.
\item
  Seleccionar ``Unidades meteorológicas''.

  \begin{enumerate}
  \def\labelenumii{\alph{enumii}.}
  \tightlist
  \item
    Sistema métrico: ºC y km/h.
  \item
    Sistema imperial: ºF y mph.
  \end{enumerate}
\end{enumerate}

\imagenAncho{meteo-info}{Información meteorológica.}{0.5
}

\subsection{Visualizar un colmenar en el
mapa}\label{visualizar-un-colmenar-en-el-mapa}

GoBees nos permite visualizar fácilmente un determinado colmenar en un
mapa utilizando nuestra aplicación de mapas favorita. De esta manera,
podemos navegar hacia él o consultar cualquier detalle cartográfico.

Video-tutorial: \url{http://gobees.io/help/videos/ver-colmenar-mapa}

Para visualizar un colmenar en el mapa:

\begin{enumerate}
\def\labelenumi{\arabic{enumi}.}
\tightlist
\item
  Entrar en el colmenar a visualizar.
\item
  Desplazarse a la pestaña ``info''.
\item
  Pulsar el botón ``mapa'' situado a la derecha de la localización del
  colmenar.
\item
  Seleccionar la aplicación con la que se desea visualizar el colmenar.
\end{enumerate}

\subsection{Añadir una colmena}\label{auxf1adir-una-colmena}

Cada colmena pertenece a un colmenar y tiene un nombre y unas notas.
Además, se puede monitorizar su actividad de vuelo, dando lugar a
grabaciones.

Video-tutorial: \url{http://gobees.io/help/videos/anadir-colmena}

Para añadir una colmena en un determinado colmenar:

\begin{enumerate}
\def\labelenumi{\arabic{enumi}.}
\tightlist
\item
  Entrar en el colmenar al que pertenecerá.
\item
  Definir el nombre de la colmena (obligatorio).
\item
  Definir unas notas sobre la colmena (opcional). En las notas se puede
  apuntar cualquier cosa relacionada con la colmena en concreto.
\item
  Pulsar el botón {$\checkmark$} para guardar la nueva colmena.
\end{enumerate}

\subsection{Editar una colmena}\label{editar-una-colmena}

Los detalles de una colmena se pueden editar en cualquier momento.

Video-tutorial: \url{http://gobees.io/help/videos/editar-colmena}

Para editar una colmena existente:

\begin{enumerate}
\def\labelenumi{\arabic{enumi}.}
\tightlist
\item
  Entrar en el colmenar al que pertenece la colmena.
\item
  Pulsar el botón de menú asociado a la colmena a editar (tres puntos
  verticales situados en la esquina superior derecha).
\item
  Seleccionar la opción de editar.
\item
  Se abrirá la pantalla de edición, donde se podrán modificar los datos
  que se deseen.
\item
  Pulsar el botón {$\checkmark$} para actualizar los datos editados.
\end{enumerate}

\subsection{Eliminar una colmena}\label{eliminar-una-colmena}

Al eliminar una colmena, se eliminan también todos los datos asociados a
esta (información de la colmena y sus grabaciones).

Video-tutorial: \url{http://gobees.io/help/videos/eliminar-colmena}

Para eliminar una colmena existente:

\begin{enumerate}
\def\labelenumi{\arabic{enumi}.}
\tightlist
\item
  Entrar en el colmenar al que pertenece la colmena.
\item
  Pulsar el botón de menú asociado a la colmena a editar (tres puntos
  verticales situados en la esquina superior derecha).
\item
  Seleccionar la opción de eliminar.
\item
  La colmena se eliminará junto con toda su información.
\end{enumerate}

\subsection{Monitorizar la actividad de vuelo de una
colmena}\label{monitorizar-la-actividad-de-vuelo-de-una-colmena}

La actividad de vuelo, junto con información previa de la colmena y
conocimiento de las condiciones locales, permite conocer al apicultor el
estado de la colmena con bastante seguridad, pudiendo determinar si esta
necesita o no una intervención.

GoBees permite monitorizar este parámetro utilizando la cámara del
\emph{smartphone}.

Video-tutorial:
\url{http://gobees.io/help/videos/monitorizacion-act-vuelo}

Para monitorizar la actividad de vuelo es necesario colocar el
\emph{smartphone} de forma fija en posición cenital a la colmena. Para
esto, se puede utilizar un trípode o un soporte similar. En la siguiente
imagen se puede ver un ejemplo de colocación:

\imagenAncho{cenital}{Colocación del \emph{smartphone} en la colmena.}{0.75}

Para mejorar los resultados de la monitorización, es recomendable que el
suelo sea de un color claro y uniforme. Si posee maleza, se puede
colocar un cartón o similar, como se muestra en la imagen.

Una vez realizado en montaje, hay que seguir los siguientes pasos dentro
de la aplicación:

\begin{enumerate}
\def\labelenumi{\arabic{enumi}.}
\tightlist
\item
  Entrar en el colmenar al que pertenece la colmena a monitorizar.
\item
  Entrar en la colmena.
\item
  Pulsar en el botón de ``monitorización'' (situado en la parte inferior
  derecha con un icono de una cámara).
\item
  Se abrirá una ventana que permite previsualizar la monitorización.
\item
  Para configurar los parámetros de la monitorización, pulsar el botón
  ``ajustes'' (situado en la parte superior derecha). Se abrirá una
  pantalla con los siguientes ajustes:

  \begin{itemize}
  \tightlist
  \item
    \textbf{Mostrar salida del algoritmo}: si no se encuentra activado
    se previsualiza la imagen proveniente de la cámara. Si se activa, se
    muestran en verde las abejas detectadas y en rojo otros objetos en
    movimiento que el algoritmo no considera abejas. Además, en la
    esquina inferior derecha se puede visualizar el número total de
    abejas contadas en cada fotograma.
  \item
    \textbf{Modificar el tamaño de las regiones}: dependiendo de la
    distancia a la que esté situada la cámara, es posible que las abejas
    se visualicen demasiado pequeñas o demasiado grandes. Con esta
    opción, se puede agrandar o disminuir su silueta.
  \item
    \textbf{Min. área abeja}: la detección de una abeja se realiza por
    área. Si el contorno en movimiento detectado posee un área dentro de
    unos límites se considera una abeja. Este parámetro configura la
    cota inferior del área. Bien ajustado, permite descartar moscas y
    mosquitos.
  \item
    \textbf{Max. área abeja}: configura la cota superior del área.
    Permite descartar la mayoría de animales que pueden habitar en el
    colmenar (avispones, roedores, lagartos o cualquier animal de mayor
    tamaño).
  \item
    \textbf{Zoom}: permite configurar el zoom de la cámara para
    encuadrar la superficie deseada.
  \item
    \textbf{Frecuencia de muestreo}: determina el intervalo de tiempo
    entre un fotograma analizado y el siguiente a analizar. Es decir, si
    se establece en 1 segundo, la aplicación captará y analizará un
    fotograma cada segundo. Cuanto mayor sea el intervalo menor será el
    consumo de batería.
  \end{itemize}
\item
  Una vez configurados los parámetros correctamente, se puede iniciar la
  monitorización pulsado el botón blanco.
\item
  Se iniciará una cuenta atrás y comenzará la monitorización. Durante
  esta, la pantalla puede estar apagada para ahorrar batería. Se puede
  aprovechar la cuenta atrás para apagarla sin influir en la
  monitorización (al manipular el móvil siempre se producen
  trepidaciones).
\item
  Cuando se desee detener la monitorización, se debe pulsar el botón
  cuadrado rojo. Una vez pulsado, se guardará la grabación y se podrá
  acceder a los detalles de esta.
\end{enumerate}

\textbf{Nota:} Si se posee alguna aplicación de ahorro de batería es imprescindible
añadir una excepción a la aplicación GoBees para que esta se pueda
ejecutar en segundo plano sin restricciones. Si no, la aplicación puede
ser cerrada durante la monitorización.

\imagen{monitoring-settings}{Ajustes de monitorización.}

\subsection{Ver los detalles de una
grabación}\label{ver-los-detalles-de-una-grabacion}

Al monitorizar una colmena se genera lo que denominamos una grabación.
Una grabación contiene los datos de actividad de vuelo de la colmena.

Video-tutorial: \url{http://gobees.io/help/videos/ver-grabacion}

Para ver los detalles de una grabación:

\begin{enumerate}
\def\labelenumi{\arabic{enumi}.}
\tightlist
\item
  Entrar en el colmenar al que pertenece la colmena monitorizada.
\item
  Entrar en la colmena.
\item
  Pulsar en la grabación sobre la que se está interesado.
\item
  Se mostrará una pantalla con dos gráficos.

  \begin{enumerate}
  \def\labelenumii{\alph{enumii}.}
  \tightlist
  \item
    El gráfico principal muestra la actividad de vuelo. En el eje de las
    Y se representa el número de abejas en vuelo y en las X los
    instantes de tiempo. Si se pulsa sobre un punto del gráfico, se
    obtiene la medida exacta en ese punto.
  \item
    El gráfico inferior muestra la información meteorológica. Existe un
    selector con tres botones: temperatura, precipitaciones y viento.
    Según se presione en uno u otro, se muestra su gráfico
    correspondiente.
  \end{enumerate}
\item
  Con ambos gráficos se puede interpretar la actividad de vuelo de la
  colmena y determinar si es una actividad normal o la colmena necesita
  una intervención.
\end{enumerate}

\imagenAncho{recording-detail}{Detalle de la grabación.}{0.5}

\subsection{Eliminar una grabación}\label{eliminar-una-grabaciuxf3n}

Al eliminar una grabación, se eliminan también todos los datos asociados
a esta.

Video-tutorial: \url{http://gobees.io/help/videos/eliminar-grabacion}

Para eliminar una grabación existente:

\begin{enumerate}
\def\labelenumi{\arabic{enumi}.}
\tightlist
\item
  Entrar en el colmenar al que pertenece la colmena monitorizada.
\item
  Entrar en la colmena.
\item
  Localizar la grabación y pulsar el botón de menú asociado a esta (tres
  puntos verticales situados en la esquina superior derecha).
\item
  Seleccionar la opción de eliminar.
\item
  La grabación se eliminará junto con toda su información.
\end{enumerate}

\subsection{Eliminar toda la información de la
aplicación}\label{eliminar-toda-la-informaciuxf3n-de-la-aplicaciuxf3n}

Si por algún motivo se desea resetear toda la información almacenada en
la aplicación, esta cuenta una opción para ello.

Video-tutorial: \url{http://gobees.io/help/videos/eliminar-datos}

Para eliminar toda la información de la aplicación:

\begin{enumerate}
\def\labelenumi{\arabic{enumi}.}
\tightlist
\item
  Pulsar el botón menú.
\item
  Entrar en la sección ``Ajustes''.
\item
  Seleccionar la opción ``Borrar todos los datos''.
\item
  Todos los datos de la aplicación serán borrados. La aplicación volverá
  al mismo estado que cuando se instaló.
\end{enumerate}

\subsection{Consultar la información sobre la
aplicación}\label{consultar-la-informaciuxf3n-sobre-la-aplicaciuxf3n}

Para conocer la versión instalada de la aplicación, los cambios
introducidos en las diferentes versiones, la licencia o el autor de esta
hay que acceder a la sección ``Acerca de GoBees''.

Video-tutorial: \url{http://gobees.io/help/videos/acerca-gobees}

Para acceder a la sección ``Acerca de GoBees'':

\begin{enumerate}
\def\labelenumi{\arabic{enumi}.}
\tightlist
\item
  Pulsar el botón menú.
\item
  Entrar en la sección ``Acerca de GoBees''.
\item
  En ella se puede visualizar la versión de la aplicación, el autor y
  las bibliotecas utilizadas para su desarrollo.
\item
  Si se presiona el botón ``Website'' se accede a la página web de
  GoBees.
\item
  Si se presiona el botón ``Licencia'' se visualiza una copia de la
  licencia de la aplicación.
\item
  Si se presiona el botón ``\emph{Changelog}'' se visualizan los cambios
  introducidos en cada versión.
\end{enumerate}

\imagenAncho{about-gobees}{Sobre GoBees.}{0.5}


\bibliography{bibliografiaAnexos}
\bibliographystyle{plainnat}

\newenvironment{bottompar}{\par\vspace*{\fill}}{\clearpage}

\begin{bottompar}
\begin{figure}[H]
	\centering
	\includegraphics[width=2cm]{ccby}
\end{figure}


\begin{center}
Este obra está bajo una licencia Creative Commons Reconocimiento 4.0 Internacional
(\href{https://creativecommons.org/licenses/by/4.0/}{CC-BY-4.0}).
\end{center}
\end{bottompar}

\end{document}
